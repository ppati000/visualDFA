\documentclass[parskip=full,11pt]{scrartcl}

\usepackage[utf8]{inputenc}
\usepackage[T1]{fontenc}
\usepackage[german]{babel}

\usepackage[yyyymmdd]{datetime} % must be after babel
\renewcommand{\dateseparator}{-} % ISO8601 date format
\usepackage{hyperref}
\usepackage{listings}
\usepackage{textcomp}

\usepackage{float}
\usepackage{enumitem}
\renewcommand{\familydefault}{\sfdefault}
\usepackage{graphicx}
\usepackage{placeins}

\usepackage{lmodern}
\usepackage{courier}

\usepackage{enumitem}
\setitemize{itemsep=-8pt}
\usepackage{csquotes}


\usepackage{todonotes} % for todo notes - usage \todo{note}

\usepackage{linegoal,listings}
\newsavebox{\mylisting}
\makeatletter
\newcommand{\lstInline}[2][,]{%
	\begingroup%
	\lstset{#1}% Set any keys locally
	\begin{lrbox}{\mylisting}\lstinline!#2!\end{lrbox}% Store listing in \mylisting
	\setlength{\@tempdima}{\linegoal}% Space left on line.
	\ifdim\wd\mylisting>\@tempdima\hfill\\\fi% Insert line break
	\lstinline!#2!% Reset listing
	\endgroup%
}
\makeatother
\setlength{\parindent}{0pt}% Just for this example

\lstset{basicstyle=\footnotesize\ttfamily,breaklines=true}
\lstset{framextopmargin=50pt,frame=bottomline,showstringspaces=false,upquote=true,tabsize=4, numbers=left}


\RedeclareSectionCommand[style=section,indent=0pt,font=\usekomafont{partnumber}]{part}
\renewcommand*{\partformat}{\thepart\enskip}

\RedeclareSectionCommand[beforeskip=0pt ,afterskip=0pt]{subparagraph}

\usepackage[bottom = 3 cm, top = 3 cm]{geometry}

\newcommand{\class}[1]{\subsubsection*{\lstinline[basicstyle=\ttfamily\large]{#1}}}

% command for an attribute
\newcommand{\attr}[4]{\lstinline{[#3]} \textbf{\lstinline{#1 : #2}} \newline #4}

% command for a method
\newcommand{\method}[5]{\lstinline{[#4]} \textbf{\lstinline{#1(#3) : #2}} \newline #5}

% command for a constructor
\newcommand{\ctor}[4]{\lstinline{[#3]} \textbf{\lstinline{#1(#2)}} \newline #4}

\newcommand{\inlinecode}[1]{\lstInline[breaklines=true]{#1}}

% make the bullet symbol in lists a circle for level 2
\renewcommand{\labelitemii}{$\circ$}

% for double quotation marks in listings
\newcommand{\qq}[1]{``#1``}

\graphicspath{ {images/} }


\begin{document}
	\title{A quick guide to the implementation of analyses}
	\date{}
	
	\maketitle
	
	\section{The API}
	The package relevant for implementing analyses is \inlinecode{dfa.framework}. 
	The important classes and interfaces are:
	
	\subsection*{LatticeElement}
	A \inlinecode{LatticeElement} represents an in- or out-state of any basic block at any given time.
	How the \inlinecode{LatticeElement} for any particular analysis looks like, entirely depends on the analysis.
	The only method a \inlinecode{LatticeElement} must provide is \inlinecode{String getStringRepresentation()}.
	This is used to create the text in the state-panel (for in- and out-states) on the right in the GUI.
	
	Although \inlinecode{getStringRepresentation()} is the only method that is mandatory to implement for a \inlinecode{LatticeElement}, great care should be taken when implementing \inlinecode{boolean} \inlinecode{equals(Object o)}. 
	Since this method is used by the \inlinecode{DFAExecution} to determine, whether two \inlinecode{LatticeElements} are equal, incorrect implementation can easily lead to infinite loops during the precalculation of a \inlinecode{DFAExecution}.
	This is due to the fact that basic blocks are added to the worklist, whenever the out-state of one of its predecessors changes (according to \inlinecode{equals}).
	
	\subsection*{\inlinecode{DFAFactory<E extends LatticeElement>}}
	A \inlinecode{DFAFactory} is used to create analyses.
	Its name is determined by \inlinecode{getName()}, this is the name shown in the analyses selection box in the GUI.
	It also determines the direction (\inlinecode{DFADirection}, either forward or backward) of the analyses created from this \inlinecode{DFAFactory}.
	Finally a \inlinecode{DFAFactory} can create instances of \inlinecode{DataFlowAnalysis} from a \inlinecode{SimpleBlockGraph} with \inlinecode{DataFlowAnalysis<E>} \inlinecode{getAnalysis(SimpleBlockGraph blockGraph)}.
	
	\subsection*{\inlinecode{DataFlowAnalysis<E extends LatticeElement>}}
	This abstract class just combines the three interfaces \inlinecode{Initializer}, \inlinecode{Join} and \inlinecode{Transition}, each of which provides only one method:
	\begin{itemize}
		\item \inlinecode{Map<Block, BlockState<E>> getInitialStates()}: \\
		This determines the initial state for each \inlinecode{soot.toolkits.graph.Block}.
		For many analyses all blocks except the start block will be initialized with some default state.
		The in-state of the start block will sometimes require special initialization.
		E. g. for constant-folding in the initial in-state of the start block the values of all locals are set to 0.
		\item \inlinecode{E join(Set<E>)}: \\
		This method is used to join several \inlinecode{LatticeElement}s into one, whenever the control flow requires it.
		The join entirely depends on the particular \inlinecode{LatticeElement} and thus on the analysis.
		But for most analyses the join-operation is rather straightforward.
		\item \inlinecode{E transition(E element, Unit unit)}
		This method decides how a \inlinecode{soot.Unit} transforms a \inlinecode{LatticeElement}.
		Usually the most effort goes into the implementation of \inlinecode{transition} because there are a lot of different \inlinecode{Unit}s that have to be handled individually.
		If you are not familiar with Soot, this can be a bit scary at first.
		In addition to looking at the API-documentation of soot you will probably want to take a look at the source code of an existing analysis.
		A good point to start would probably be \inlinecode{ConstantFoldingTransition} since it is a functioning analysis and is still reasonably straightforward.
	\end{itemize}

	One last tip that may be helpful: when you encounter a \inlinecode{Unit} you don't want to handle, you can simply throw a subclass of \inlinecode{DFAException} (e. g. \inlinecode{UnsupportedStatementException}).
	In that case the precalculation of a \inlinecode{DFAExecution} will be aborted in an orderly fashion and an error message is shown in the GUI.
	But this also means, you can't analyze any code that produces those \inlinecode{Unit}s.
	There may also be cases, where you can safely ignore a particular \inlinecode{Unit}.
	For examples on this see the code of \inlinecode{ConstantFoldingTransition}.

	\subsection*{\inlinecode{CompositeDataFlowAnalysis<E extends LatticeElement>}}
	This is just a subclass of \inlinecode{DataFlowAnalysis} which takes one \inlinecode{Initializer}, one \inlinecode{Join} and one \inlinecode{Transition} in its constructor and delegates the calls \inlinecode{getInitialStates(...)}, \inlinecode{join(...)} and \inlinecode{transition(...)} to the given objects.
	So there are two ways of implementing a new analysis: You can extends \inlinecode{DataFlowAnalysis} and overwrite the three methods or you can implement three classes each of which implements one of the three interfaces and then just create a \inlinecode{CompositeDataFlowAnalysis} from them.
	Which one you choose is up to you, it doesn't really make a difference.
	We prefer the second approach since it naturally structures the analysis into several classes.
	If you chose to just implement the three methods in one class you will probably end up with a rather lengthy class.

	\clearpage

	\section{A minimal example}
	Previously the interfaces needed to implement a new analysis were introduced.
	Now a minimal example on how to implement an analysis is given.
	This example is not a useful analysis by any means.
	It only shows which methods must be implemented and how it could be done.
	For an example of a real analysis you could look at \inlinecode{ConstantFoldingAnalysis} (with all the classes used by it, especially \inlinecode{ConstantFoldingTransition}).
	
	\subsection*{\inlinecode{DFAFactory}}
	\begin{lstlisting}
package dfa.analyses.testanalyses;

import dfa.framework.DFADirection;
import dfa.framework.DFAFactory;
import dfa.framework.DataFlowAnalysis;
import dfa.framework.SimpleBlockGraph;

public class DummyFactory extends DFAFactory<DummyElement> {

	@Override
	public String getName() {
		// the name shown in the analysis selction
		return "Dummy Analysis"; 
	}

	@Override
	public DFADirection getDirection() {
		// let's make it a forwrad analysis
		return DFADirection.FORWARD;
	}

	@Override
	public DataFlowAnalysis<DummyElement> getAnalysis(SimpleBlockGraph blockGraph) {
		return new DummyAnalysis(blockGraph);
	}
}

	\end{lstlisting}
	
	\subsection*{\inlinecode{[Composite-]DataFlowAnalysis}}
	\begin{lstlisting}
package dfa.analyses.testanalyses;

import dfa.framework.CompositeDataFlowAnalysis;
import dfa.framework.SimpleBlockGraph;

public class DummyAnalysis extends CompositeDataFlowAnalysis<DummyElement> {
	public DummyAnalysis(SimpleBlockGraph blockGraph) {
		// since we extend a CompositeDataFlowAnalysis we use 
		// its constructor and are done
		super(new DummyJoin(), new DummyTransition(), new DummyInitializer(blockGraph));
	}
}
	\end{lstlisting}
	
	\clearpage
	
	\subsection*{\inlinecode{LatticeElement}}
	\begin{lstlisting}
package dfa.analyses.testanalyses;

import dfa.framework.LatticeElement;

public class DummyElement implements LatticeElement {
	public final ValueType type;

	public DummyElement(ValueType type) {
		this.type = type;
	}

	@Override
	public boolean equals(Object o) {
		if (!(o instanceof DummyElement)) {
			return false;
		}

		DummyElement e = (DummyElement) o;
		
		// we make sure, that the same types are 
		// treated as equal DummyElements
		return e.type == this.type;
	}

	@Override
	public String getStringRepresentation() {
		// here we craft the string representation of an element, this will be shown in the GUI
		switch (type) {
		case BOTTOM:
			return "B";
		case TOP:
			return "T";
		case SOMETHING:
			return "?";
		default:
			throw new IllegalStateException();
		}
	}

	enum ValueType {
		BOTTOM, TOP, SOMETHING
	}
}
	\end{lstlisting}
	
\clearpage
	
	\subsection*{\inlinecode{Initializer}}
	\begin{lstlisting}
package dfa.analyses.testanalyses;

import dfa.framework.BlockState;
import dfa.framework.Initializer;
import dfa.framework.SimpleBlockGraph;
import soot.toolkits.graph.Block;

import java.util.HashMap;
import java.util.List;
import java.util.Map;

public class DummyInitializer implements Initializer<DummyElement> {
	private SimpleBlockGraph blockGraph;
	
	public DummyInitializer(SimpleBlockGraph blockGraph) {
		this.blockGraph = blockGraph;
	}
	
	// this method will be used by the DFAExecution to initialize
	// the in- and out-states of all blocks
	@Override
	public Map<Block, BlockState<DummyElement>> getInitialStates() {
		List<Block> blocks = blockGraph.getBlocks();
	
		BlockState<DummyElement> defaultState = new BlockState<>(new DummyElement(DummyElement.ValueType.TOP), new DummyElement(DummyElement.ValueType.BOTTOM));
	
		Map<Block, BlockState<DummyElement>> initialMap = new HashMap<>();
	
		// each block simply gets the same initial in- and out-state
		for (Block b : blocks) {
			initialMap.put(b, defaultState);
		}

		return initialMap;
	}
}
	\end{lstlisting}
	
	\clearpage
	
	\subsection*{\inlinecode{Join}}
	\begin{lstlisting}
package dfa.analyses.testanalyses;

import dfa.framework.Join;

import java.util.Set;

public class DummyJoin implements Join<DummyElement> {

	// this method will be called whenever the (one ore more) 
	// predecessors of one block need to be joined
	@Override
	public DummyElement join(Set<DummyElement> elements) {
		// a trivial join, for real analyses you will have to do a little more work
		return new DummyElement(DummyElement.ValueType.SOMETHING);
	}
}
	\end{lstlisting}	

	\subsection*{\inlinecode{Transition}}
	\begin{lstlisting}
package dfa.analyses.testanalyses;

import dfa.framework.Transition;
import soot.Unit;

public class DummyTransition implements Transition<DummyElement> {

	// this method will be called to transition one state to another 
	// over the given Unit
	@Override
	public DummyElement transition(DummyElement element, Unit unit) {
		// a trivial transition, for real analyses you will usually have to do
		// quite some work here (and at least respect the parameter unit)
		return new DummyElement(DummyElement.ValueType.SOMETHING);
	}
}
	\end{lstlisting}	

\end{document}