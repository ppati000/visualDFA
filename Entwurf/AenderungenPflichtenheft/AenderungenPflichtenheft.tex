\part{Nachträgliche Änderungen am Pflichtenheft}

\subsection*{Unterstütztes Java-Subset} 
Da jede Analyse mit (Jimple-)Statements auf eine eigene Art und Weise umgeht, ist in jeder implementierten Analyse der Umgang mit den Statements einzeln definiert.
Dies bedeutet, dass alle im Pflichtenheft angegebenen Kontrollflussstrukturen und Statements für alle im Pflichtenheft angegebenen Analysen implementiert werden.
Falls ein Endbenutzer sich dazu entschließt selbst eine Analyse zu erstellen, so muss er selbst festlegen mit welchen Statements seine Analyse funktionieren soll und den Umgang mit diesen implementieren.
Stößt das Programm bei der Durchführung der Analyse auf ein Statement für welches kein Verhalten festgelegt wurde, so wird eine Fehlermeldung erzeugt.