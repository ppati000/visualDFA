%note: don't split this document up with include{...}

\section{CodeProcessor}

Das CodeProcessor-Modul ist dafür verantwortlich den eingegebenen Code des Benutzers in Java-Bytecode und letztendlich in einen Kontrollflussgraphen umzuwandeln, welcher in der weiteren Programmlogik verwendet werden kann.

Es gibt zwei Schnittstellen; CodeProcessor und GraphBuilder.
Der CodeProcessor erhält vom Controller den eingegebenen Code des Benutzers und versucht diesen zu Java-Bytecode zu kompilieren.

Ist dies erfolgreich, so ruft der Controller den Graphbuilder auf und lässt den Java-Bytecode mithilfe der externen Bibliothek Soot in eine Zwischenrepräsentation umwandeln.
Aus dieser können dann alle Methoden ausgelesen werden, welche in dem zu analysierenden Code enthalten sind.
Aus diesen Methoden können über das Interface Filter, beziehungsweise seine Implementierungen unnötige Methoden, wie zum Beispiel geerbte Methoden herausgefiltert werden.
Abschließend kann man dem GraphBuilder die Signatur der zu analysierenden Methode übergeben und erhält einen BlockGraph zu der entsprechenden Methode, welcher in der weiteren Programmlogik verwendet werden kann.