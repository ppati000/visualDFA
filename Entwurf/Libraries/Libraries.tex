%note: don't split this document up with include{...}

\section{Verwendete Libraries}

\subsection{JGraphX}

JGraphX (\url{https://github.com/jgraph/jgraphx}) ist eine Java-Libraray zum automatischen Layouten und zum Zeichnen von Graphen. 
Damit eignet sich JGraphX gut, um den im Rahmen der Animation von Datenflussanalysen erstellten Kontrollflussgraphen zu visualisieren.
Insbesondere bietet JGraphX die Möglichkeit, Knoten weiter in Zellen (\lstinline{mxCell}) zu unterteilen.
Dies ermöglicht die zeilenweise Animation der Datenflussanalysen.

\subsection{Reflections}

Reflections (\url{https://github.com/ronmamo/reflections}) ist eine Java-Library, die es erlaubt, zur Laufzeit neue Java-Klassen zu laden sowie Metainformation über diese zu erhalten.
Dies wird dazu benutzt, um zur Laufzeit neue Datenflussanalysen zu laden.
Dazu wird ein festgelegter Ordner bei Programmstart auf kompilierte Java-Dateien (.class-Dateien) aus einem bestimmten Package [?! noch festlegen !?] untersucht. 
Dann werden alle Klassen, die von \lstinline{DataFlowExecution} erben geladen und als Datenflussanalysen bereitgestellt. 


\subsection{Soot}

Soot (\url{https://github.com/Sable/soot}) ist eine Java-Library, die viele Funktionalitäten für statische Programmanalysen bereitstellt. 
Insbesondere bietet Soot mehrere Zwischencode-Formate an, in die gegebener Java-Bytecode übersetzt werden kann. 
In diesem Projekt wird Jimple als Zwischencode verwendet, auf dem die Datenflussanalysen erfolgen.
Bei Jimple handelt es sich um einen Expression-basierten typisierte Drei-Adress-Code.
Die elementaren Codeeinheiten sind also Expressions (Ausdrücke), die jeweils maximal drei Operanden haben.
Neben einer Zwischencode-Repräsentation bietet Soot die Möglichkeit, einen Kontrollflussgraphen aus Java-Bytecode zu erzeugen.
Die Grundblöcke des erzeugten Kontrollflussgraphen enthalten dann entsprechenden Zwischencode (hier Jimple).
