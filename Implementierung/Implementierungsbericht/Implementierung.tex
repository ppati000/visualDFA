\documentclass[parskip=full,11pt]{scrartcl}

\usepackage[utf8]{inputenc}
\usepackage[T1]{fontenc}
\usepackage[german]{babel}

\usepackage[yyyymmdd]{datetime} % must be after babel
\renewcommand{\dateseparator}{-} % ISO8601 date format
\usepackage{hyperref}
\usepackage{listings}
\usepackage{textcomp}

\usepackage{float}
\usepackage{enumitem}
\renewcommand{\familydefault}{\sfdefault}
\usepackage{graphicx}
\usepackage{placeins}

\usepackage{lmodern}
\usepackage{courier}

\usepackage{enumitem}
\setitemize{itemsep=-8pt}
\usepackage{csquotes}


\usepackage{todonotes} % for todo notes - usage \todo{note}

\usepackage{linegoal,listings}
\newsavebox{\mylisting}
\makeatletter
\newcommand{\lstInline}[2][,]{%
	\begingroup%
	\lstset{#1}% Set any keys locally
	\begin{lrbox}{\mylisting}\lstinline!#2!\end{lrbox}% Store listing in \mylisting
	\setlength{\@tempdima}{\linegoal}% Space left on line.
	\ifdim\wd\mylisting>\@tempdima\hfill\\\fi% Insert line break
	\lstinline!#2!% Reset listing
	\endgroup%
}
\makeatother
\setlength{\parindent}{0pt}% Just for this example

\lstset{basicstyle=\footnotesize\ttfamily,breaklines=true}
\lstset{framextopmargin=50pt,frame=bottomline,showstringspaces=false,upquote=true}


\RedeclareSectionCommand[style=section,indent=0pt,font=\usekomafont{partnumber}]{part}
\renewcommand*{\partformat}{\thepart\enskip}

\RedeclareSectionCommand[beforeskip=0pt ,afterskip=0pt]{subparagraph}

\usepackage[bottom = 3 cm, top = 3 cm]{geometry}

\newcommand{\class}[1]{\subsubsection*{\lstinline[basicstyle=\ttfamily\large]{#1}}}

% command for an attribute
\newcommand{\attr}[4]{\lstinline{[#3]} \textbf{\lstinline{#1 : #2}} \newline #4}

% command for a method
\newcommand{\method}[5]{\lstinline{[#4]} \textbf{\lstinline{#1(#3) : #2}} \newline #5}

% command for a constructor
\newcommand{\ctor}[4]{\lstinline{[#3]} \textbf{\lstinline{#1(#2)}} \newline #4}

\newcommand{\inlinecode}[1]{\lstInline[breaklines=true]{#1}}

% make the bullet symbol in lists a circle for level 2
\renewcommand{\labelitemii}{$\circ$}

% for double quotation marks in listings
\newcommand{\qq}[1]{``#1``}

\graphicspath{ {images/} }


\begin{document}
	\title{Implementierung: Programmanalyse zum Durchklicken}
	\author{Nils Jessen \and Anika Nietzer \and Patrick Petrovic \and Sebastian Rauch \and Michael Schieber}
	
	\maketitle

    %note: don't split this document up with include{...}

\section{Einleitung}

Moderne Compiler übersetzen nicht nur Quellcode, sondern bieten auch die Möglichkeit, Optimierungen durchzuführen. 
Einige dieser Optimierungen basieren auf Datenflussanalysen des Quellcodes bzw. eines Zwischencodes.
Um das Verständnis für auf Datenflussanalyse basierende Optimierungen zu erleichtern, soll hier ein Werkzeug entwickelt werden, welches ausgewählte Datenflussanalysen visualisiert.

\subsection{Kontrollflussgraph}
Das Konzept des Kontrollflussgraphen (CFG) ist zentral für Datenflussanalysen und wird deshalb im Folgenden kurz erklärt.
Ein CFG zu einem Programm (oder einer Funktion / Methode) ist ein Graph, dessen Knoten genau die Grundblöcke des Programms sind.
Ein Grundblock $B$ ist eine maximal lange Folge von Instruktionen, sodass der Kontrollfluss $B$ an höchstens einer Stelle betreten und an höchstens einer Stelle verlassen kann.
Es gibt also im Programm keine Sprünge, deren Ziel nicht Anfang eines Grundblockes ist.
Außerdem stehen Sprungbefehle höchstens am Ende eines Grundblocks.
Zwischen zwei Grundblöcken $B_i$ und $B_j$ gibt es genau dann eine gerichtete Kante $(B_i,B_j)$, wenn der Kontrollfluss unmittelbar von $B_i$ nach $B_j$ wechseln kann.
In diesem Fall heißt $B_i$ Vorgänger von $B_j$ und $B_j$ Nachfolger von $B_i$.
Dies ist z. B. der Fall, wenn am Ende von $B_i$ eine Sprunganweisung steht, deren Ziel der Anfang von $B_j$ ist.

\par

\begin{lstlisting}[frame=single, captionpos=b, caption=Simple Funktion zur Veranschaulichung eines CFG]
int gcd(int a, int b) {
	int tmp;
	while(b != 0) {
		tmp = b;
		b = a % b;
		a = tmp;
	}
	return a;
}
\end{lstlisting}

\par

\begin{figure}[H]
\centering
\begin{tikzpicture}[%
->,
shorten >=2pt,
>=stealth,
node distance=1cm,
noname/.style={%
	ellipse,
	minimum width=5em,
	minimum height=3em,
	draw
}
]
\node [draw] (1) {
\begin{lstlisting}[numbers=none]
int tmp;
\end{lstlisting}
};

\node [draw] (2) [below=of 1] {
\begin{lstlisting}[numbers=none]
if (b != 0)
\end{lstlisting}
};

\node[draw] (3) [below left=of 2]   {
\begin{lstlisting}[numbers=none]
tmp = b;
b = a % b;
a = tmp;
\end{lstlisting}
};

\node[draw] (4) [below right=of 2]   {
\begin{lstlisting}[numbers=none]
return a;
\end{lstlisting}
};

\path (1) edge node {} (2);
\path (2) edge [bend right=20pt] node {} (3);
\path (2) edge node {} (4);
\path (3) edge [bend right=20pt] node {} (2);
\end{tikzpicture}
\caption{CFG zu Listing 1}
\end{figure}

\par

\subsection{Datenflussanalyse}
Eine Datenflussanalyse ist eine statische Code-Analyse, d. h. die Analyse erfolgt lediglich anhand der Struktur des Codes, ohne dass dieser ausgeführt wird. 
Mittels Datenflussanalyse können bestimmte Eigenschaften von Programmen approximiert werden.

\subsubsection{Theoretische Grundlagen}

Datenflussanalysen arbeiten typischerweise auf dem CFG des zu analysierenden Codes.
Im Folgenden werden Grundblöcke als kleinste Einheit betrachtet, auf denen die Analyse operiert.
Für eine Datenflussanalyse benötigt man zunächst eine Grundmenge $D$.
Diese Grundmenge stellt die möglichen Fakten dar, die vor bzw. nach einem Grundblock gelten können.
Zu jedem diskreten Schritt $t$ der Analyse ist jedem Grundblock $B$ ein Eingangszustand $in_B[t] \in D$ und ein Austrittszustand $out_B[t] \in D$ zugeordnet.
Der Eingangszustand von $B$ repräsentiert die bei Eintritt in $B$ geltenden Fakten.
Analog entspricht der Austrittszustand von $B$ den geltenden Fakten bei Verlassen von $B$.

Weiter ist zu jedem Grundblock $B$ eine Überführungsfunktion $f_B:D\to D$ gegeben, die den Austrittszustand in Schritt $t+1$ aus dem Eingangszustand im vorherigen Schritt $t$ berechnet: $out_B[t+1]=f_B(in_B[t])$.
Diese Überführungsfunktion setzt sich aus den Überführungsfunktionen der einzelnen Instruktionen innerhalb dieses Grundblocks zusammen:
Besteht der Grundblock $B$ aus den Instruktionen $I_1,...,I_m$ in dieser Reihenfolge und ist für jede dieser Instruktionen eine Überführungsfunktion $f_{I_k}:D\to D, k\in \{1,..,m\}$ gegeben, so ist $f_B$ als Komposition dieser gegeben: $f_B=f_{I_m}\circ ... \circ f_{I_1}$.
Deshalb genügt es, zu jeder Instruktion eine Überführungsfunktion zu definieren, die die Wirkung dieser Instruktion beschreibt.

Ziel einer Datenflussanalyse ist nun, für jeden Grundblock $B$ ein Faktum $d\in D$ zu finden, das nach $B$ gilt und sich bei obiger Iteration ($out_B[t+1]=f_B(in_B[t])$) nicht mehr ändert.
Dies kann mittels Fixpunktiteration erreicht werden.
Dazu benötigt man zusätzlich zu den Überführungsfunktionen innerhalb der Grundblöcke noch Überführungsfunktionen zwischen den Grundblöcken.
Im Allgemeinen kann ein Grundblock mehrere Vorgänger haben.
Deshalb benötigt man eine Operation, um die Austrittszustände mehrerer Grundblöcke zusammenzuführen.
Diese Operation wird im Folgenden als Join-Operator $\vee :D\times D \to D$ bezeichnet.
Ausgehend davon kann man den Join-Operator für endlich viele $d_1,...,d_n \in D$ definieren: $\bigvee (d_1,..,d_n):=(\bigvee(d_1,...,d_{n-1})) \vee d_n$ für $n \geq 3$ und $\bigvee(d_1,d_2):=d_1 \vee d_2$ (formal kann man noch $\bigvee (d_1):=d_1$ setzen).
Damit kann man Überführungsfunktionen zwischen Grundblöcken definieren: 
Hat der Grundblock $B$ die Vorgänger $P_1,...,P_r$, so ist der Eingangszustand von $B$ im Schritt $t+1$ gegeben als Join der Austrittszustände von $P_1,...,P_r$ im Schritt $t$: $in_B[t+1]=\bigvee(out_{P_1}[t],...,out_{P_r}[t])$.

Bei der Fixpunktiteration wird für $t \in \mathbb{N}$ und jeden Grundblock $B$ der Austrittszustand aus dem Eingangszustand berechnet: $out_B[t+1]=f_B(in_B[t])$. 
Dies wird fortgesetzt, bis erstmals für ein $t_0 \in \mathbb{N}$ für jeden Grundblock $B$ der Austrittszustand konstant bleibt: $out_B[t_0+1]=f_B(in_B[t_0])$.
Damit gilt wegen der Rechtseindeutigkeit der Überführungsfunktionen für die Grundblöcke $f_{B_i}$ sowie der Überführungsfunktionen zwischen den Grundblöcken: $out_B[s+k]=f_B(in_B[s])$ für alle $s,k \in \mathbb{N}$ mit $s \geq t_0$.
Es ändern sich also bei obiger Iteration die Eingangszustände und Ausgangszustände nach Schritt $t_0$ nicht mehr.
Dies bezeichnet man als Fixpunkt des hier beschriebenen iterativen Verfahrens.

Stellt man geeignete Bedingungen an die Grundmenge $D$, die Überführungsfunktion $f_B$ für jeden Grundblock $B$ sowie den Join-Operator $\vee$, so existiert solch ein Fixpunkt notwendigerweise und wird durch obiges Verfahren stets nach endlich vielen Schritten gefunden.


Im Folgenden werden einige Beispiele für Datenflussanalysen und die durch sie approximierten Eigenschaften angegeben.

\subsubsection{Constant Folding}
Beim Constant Folding werden die Werte von Ausdrücken, die bereits bei der Übersetzung ausgewertet werden können, ermittelt.
Eine naheliegende Optimierung ist dann, solche Ausdrücke durch die entsprechenden Werte zu ersetzen.
Damit vermeidet man das Generieren der Instruktionen für diese Ausdrücke. Lediglich eine Instruktion zum Laden einer Konstanten muss dann generiert werden.
\begin{lstlisting}[frame=single, captionpos=b, caption=Beispielcode für Constant-Folding-Analyse]
int x = 2;
int y = -4;
int z = x * x + y * y; 
\end{lstlisting}
Hier steht der Wert von \lstinline{z} bereits bei der Übersetzung fest:
\begin{lstlisting}[numbers=none]
z = 2 * 2 + (-4) * (-4) = 4 + 16 = 20.
\end{lstlisting}
Obiger Code ist also äquivalent zu
\begin{lstlisting}[frame=single, captionpos=b, caption=Mittels Constant-Folding optimierte Version von Listing 2]
int x = 2;
int y = -4;
int z = 20; 
\end{lstlisting}
Hier tauscht man im Vergleich zum ursprünglichen Code zwei Multiplikationen und eine Addition gegen das Laden einer Konstanten ein.

\subsubsection{Constant Bits}
Die Constant-Bits-Analyse ist eng mit dem Constant Folding verwandt. 
Hier interessiert man sich allerdings dafür, welche einzelnen Bits von Variablen konstant sind.
Sind konstante Bits innerhalb von Variablen gefunden, kann diese Information zur Optimierung bestimmter Ausdrücke genutzt werden.

\par

\begin{lstlisting}[frame=single, captionpos=b, caption=Beispielcode für Constant-Bits-Analyse]
int foo(int x) {
	int y = 12 + 8 * x;
	return y % 4 + 2 * x;
} 
\end{lstlisting}

Hier ist \lstinline|y| \textbf{\%} \lstinline|4| immer gleich \lstinline{0}, sodass obiger Code vereinfacht werden kann zu

\par

\begin{lstlisting}[frame=single, captionpos=b, caption=Mittels Constant-Bits-Analyse optimierte Version von Listing 4]
int foo(int x) {
	int y = 12 + 8 * x;
	return 2 * x;
} 
\end{lstlisting}

\subsubsection{Reaching Definitions}
Bei der Reaching-Definitions-Analyse soll für jede Stelle im Programm die Menge der Definitionen (Zuweisungen), die diese Programmstelle erreichen, ermittelt werden.
Eine Definition erreicht eine Stelle genau dann, wenn es einen Pfad im CFG gibt, sodass die Definition auf diesem Pfad vor besagter Stelle auftaucht und nicht wieder überschrieben wird.

\par

\begin{lstlisting}[frame=single, captionpos=b, caption=Beispielcode für Reaching-Definitions-Analyse]
$def_1$: int x = 2;
$def_2$: int y = -4;
$def_3$: x = 2 * y + 10; 
\end{lstlisting}

\par

Betrachtet man das Ende des obigen Codestücks, so sind $def_2$ und $def_3$ Reaching Definitions bezüglich dieser Programmstelle.
$def_1$ erreicht das Ende des Programms nicht, da $def_3$ eine neue Definition von \lstinline{x} gibt, $def_1$ ist danach nicht mehr relevant.

\subsubsection{Taint-Analyse}
In manchen Programmen ist es fatal, wenn Benutzereingaben ohne weitere Überprüfung verarbeitet werden. 
Beispielsweise sollte niemals ein String, der ganz oder teilweise von einem Benutzer stammt, direkt als Anfrage an ein Datenbanksystem weitergegeben werden.
Dies ermöglicht Angriffe wie SQL-Injection, die das Datenbanksystem in ungewollter Weise beeinflussen.
Um diese Angriffe zu vermeiden, sollten alle Anfragen, die aus Benutzereingaben hervorgehen, auf möglicherweise schädliche Effekte untersucht werden.
Das Überprüfen, ob jedes Datenobjekt, das als potentiell schädlich eingestuft wird, vor Weitergabe an kritische Stellen (z. B. Anfrageschnittstelle eines Datenbanksystems) überprüft wurde, kann mittels Taint-Analyse automatisiert geschehen.
Dazu werden Stellen im Programm definiert, die Daten als 'bedenklich' (tainted) markieren.
Weiter gibt es Stellen (z. B. spezielle Methodenaufrufe), die 'bedenkliche' Daten als 'unbedenklich' markieren.
Schließlich gibt es noch Programmstellen, die als kritisch markiert sind.
Ziel der Taint-Analyse ist es nun, sicherzustellen, dass niemals als 'bedenklich' markierte Daten eine als kritisch markierte Programmstelle erreichen.

\subsubsection{Live Variables}
Bei der Live-Variables-Analyse interessiert man sich dafür, ob der Wert einer Variablen vor der nächsten Zuweisung an diese (bzw. vor Programmende, falls keine nächste Zuweisung an diese Variable existiert) benötigt wird. 
Ist dies nicht der Fall, kann diese Zuweisung entfernt werden.

\par

\begin{lstlisting}[frame=single, captionpos=b, caption=Beispielcode für Live-Variables-Analyse]
int incMin(int x, int y) {
	int min;
	if (x < y) {
		min = x;
		x = x + 1;
	} else {
		min = y;
		y = y + 1;
	}
	return min + 1;
}
\end{lstlisting}

\par

Hier haben die beiden Zuweisungen \lstinline{x = x + 1;} und \lstinline{y = y + 1} keinen Effekt und können daher entfernt werden.
Folgender optimierter Code ist also äquivalent:

\par

\begin{lstlisting}[frame=single, captionpos=b, caption=Mittels Live-Variables-Analyse optimierte Version von Listing 7]
int incMin(int x, int y) {
	int min;
	if (x < y) {
		min = x;
	} else {
		min = y;
	}
	return min + 1;
}
\end{lstlisting}

\subsection{Zielsetzung}
Wie bereits zu Anfang formuliert, soll hier ein Werkzeug zur Visualisierung ausgewählter Datenflussanalysen entwickelt werden.
Dazu kann der Benutzer Java-ähnlichen Code (eine Teilmenge der primitiven Datentypen sowie Operatoren und Kontrollflusskonstrukte von Java werden unterstützt) zur Analyse angeben.
Aus diesem Code wird zunächst ein CFG generiert und dem Benutzer angezeigt.
Dann kann der Benutzer die zu visualisierende Datenflussanalyse auswählen.
Es werden mindestens Constant-Folding-, Constant-Bits-, Reaching-Definitions- und Taint-Analyse unterstützt.
Die ausgewählte Datenflussanalyse wird dann ausgeführt und schrittweise anhand des CFG visualisiert.
Zusätzlich hat der Benutzer noch die Möglichkeit, den Fixpunkt der Analyse direkt berechnen zu lassen.

	\part{Implementierungsplan}

\section{Aufgabenteilung}

Das Programm ist in sechs Hauptmodule aufgeteilt:

\begin{itemize}
  \item Modul \inlinecode{dfa.framework}: Stellt die nötige Infrastruktur zur Ausführung von Datenflussanalysen bereit und bietet eine Schnittstelle, um konkrete Analysen zu implementieren.
  \item Modul \inlinecode{dfa.analyses}: Implementiert die Schnittstellen des DFA-Frameworks und enthält die Logik der konkreten Analysen.
  \item Modul \inlinecode{controller}: Enthält die Startklasse und steuert das Zusammenspiel von DFA-Framework, GUI und visuellem Graphen.
  \item Modul \inlinecode{codeprocessor}: Verarbeitet die Code-Eingabe des Benutzers zu einem logischen Kontrollflussgraphen, der für die Analyse verwendet wird.
  \item Modul \inlinecode{gui}: Beinhaltet das User Interface (exkl. des visuellen Graphen)
  \item Modul \inlinecode{gui.visualgraph}: Zeigt den visuellen Graphen an und ermöglicht Benutzerinteraktion mit diesem.
\end{itemize}

Um einen kontinuierlichen Arbeitsablauf zu gewährleisten und das Auftreten von Merge Conflicts zu minimieren, wurde eine feste Aufgabenteilung festgelegt. Die Implementierung wurde unter folgender Aufteilung durchgeführt:

\begin{tabular}{l*{6}{c}r}
Modul & Person(en) \\
\hline
\inlinecode{dfa.framework} & Nils, Sebastian \\
\inlinecode{dfa.analyses} & Nils, Sebastian \\
\inlinecode{controller} & Anika \\
\inlinecode{codeprocessor} & Anika \\
\inlinecode{gui} & Michael \\
\inlinecode{gui.visualgraph} & Patrick \\
\end{tabular}

\newpage
\section{Implementierungsreihenfolge}

Es waren einige Abhängigkeiten zwischen den Modulen zu beachten. Daher wurde ein grober Ablaufplan erstellt:

\begin{enumerate}
  \item Implementierung des DFA-Frameworks sowie einer Dummy-Analyse, um schnelles Testen zu ermöglichen; parallel dazu Implementierung der Grundfunktionalität des visuellen Graphen
  \item Implementierung von Code-Processor, Controller sowie dem für den Start und die Steuerung der Analyse notwendigen Teil der GUI
  \item Restimplementierung von GUI und visuellem Graphen; parallel dazu Implementierung der konkreten Analysen.
\end{enumerate}

Die Implementierung der konkreten Analysen sollte weitgehend entkoppelt vom Rest des Programms ablaufen, da durch die Existenz der Dummy-Analyse kein anderer Teil des Programms davon abhängt.
Auch der visuelle Graph musste nicht notwendigerweise zu Beginn der Implementierung vollständig sein, da nur die Schnittstelle des \inlinecode{GraphUIController} und die Einbindung des \inlinecode{VisualGraphPanel} für die Funktion des Programms essenziell sind.
Ebenfalls nicht essenziell für die Programmfunktionalität zu Beginn der Implementierung war das \inlinecode{StatePanelOpen}, da dieses nur Zustände der Analyse anzeigt, selbst aber keine Programmlogik enthält.


	%note: don't split this document up with include{...}

\part{Umsetzung}

In diesem Kapitel wird erklärt, welche Funktionalität aus dem Pflichtenheft implementiert wurde und in welchem Ausmaß der Implementierungsplan erfüllt wurde.
Insbesondere wird dabei auf die Behandlung von unerwarteten Problemen während der Implementierung eingegangen.

\section{Umsetzung der Kriterien}
% TODO: Update if needed, e.g. Reaching Definitions. (Slides as well)

\subsection{Muss-Kriterien}

Die Muss-Kriterien wurden planmäßig und vollständig implementiert. Diese sind:

\begin{itemize}
  \item Eingabe von Code und Verarbeitung zu einem CFG
  \item Unterstützung des definierten Java-Subsets
  \item Anzeige des CFG
  \item Ausführung und Anzeige der implementierten Analysen
  \item Automatische Fixpunktiteration
  \item Abbruch der Analyse 
\end{itemize}

Insbesondere wurden alle in den Muss-Kriterien geforderten Analysen implementiert.
Diese sind Constant Folding, Reaching Definitions, Constant Bits und Taint Analysis.

\subsection{Kann-Kriterien}
% TODO: Update if needed, e.g. Reaching Definitions. (Slides as well)

Die Kann-Kriterien wurden zum Großteil implementiert.
Von insgesamt 19 Kann-Kriterien wurden 15 vollständig implementiert, ein weiteres (Auswahlmöglichkeit der Worklistalgorithmen) teilweise.
Lediglich drei Kann-Kriterien wurden \textit{nicht} implementiert. Dazu gehören:

\begin{itemize}
  \item Zeilenangabe bei Fehlermeldungen
  \item Weitere Datenflussanalysen (Live Variables, Available Expressions, Definite Assignment)
  \item Manual Page
\end{itemize}

\newpage
\section{Problembehandlung}

Es traten einige unvorhersehbare Probleme auf, die eine Änderung der geplanten Implementierung erforderten.

\subsection{Dynamisches Laden von Klassen für eigene Datenflussanalysen}

Es war vorgesehen, zum dynamischen Laden von Klassen für eigene Datenflussanalysen des Benutzers die Reflections-Bibliothek zu verwenden.
Diese hatte in einem Prototyp einwandfrei funktioniert.
Im Laufe der Implementierung stellte sich jedoch heraus, dass die Reflections-Bibliothek im Zusammenspiel mit der für Datenflussanalysen verwendeten Soot-Bibliothek einen Abhängigkeitskonflikt verursacht.
Dies kommt dadurch zustande, dass die beiden Bibliotheken unterschiedliche Versionen der selben Abhängigkeit benötigen. 
Daraus resultiert ein Laufzeitfehler, der nicht trivial behebbar ist.

Wegen des großen erwarteten Aufwands wurde ausgeschlossen, eine der Bibliotheken anzupassen. 
Es wurde entschieden, andere (insbesondere Muss-) Kriterien zuerst zu implementieren und dynamisches Laden von Klassen nur bei ausreichender zeitlicher Kapazität selbst zu implementieren.
Zu diesem Zeitpunkt lassen sich daher im Programm keine Klassen dynamisch laden.

Allerdings muss hervorgehoben werden, dass die Entwurfsarchitektur des Programmes darauf ausgelegt ist, dass jederzeit neue Datenflussanalysen implementiert werden können und das Fehlen der dynamischen Funktionalität kein großes Hindernis darstellt.
Lediglich der letzte Schritt zur Implementierung eigener Analysen fehlt also – dies lässt sich durch Rekompilieren des Programms umgehen.
Darüber hinaus kann dynamisches Laden von Klassen nachgereicht werden, ohne nennenswerte Teile des Programms umschreiben zu müssen.

Daher wird das Kriterium \enquote{Implementierung einer eigenen Datenflussanalyse} als \textit{vollständig implementiert} betrachtet.

\newpage
\subsection{Graph-Export}

Bereits vor Beginn der Implementierungsphase wurde sichergestellt, dass eine benutzbare API für den Graph-Export in der JGraphX-Bibliothek zur Verfügung steht.
Um auch den Zustand der Analyse im Bild anzeigen zu können, sollte das gesamte \inlinecode{StatePanelOpen} dort hineingerendert werden.
Dies erwies sich allerdings als fehlerbehaftet und unzuverlässig. 
Insbesondere trat das Problem auf, dass einzelne Komponenten des Panels im Bild nicht angezeigt wurden.
Darüber hinaus ließ die Qualität des Font-Renderings derart zu wünschen übrig, dass entschieden wurde, das \inlinecode{StatePanelOpen} nicht für den Graph-Export zu verwenden.

Stattdessen wurde eine Ersatzimplementierung in das Programm eingebaut, welches den Text des Analyse-Zustands mit Hilfe der Swing-Klasse \inlinecode{Graphics2D} manuell in das Bild einfügt.
Dies verursachte zwar etwas Verzögerung, erwies sich jedoch als akzeptable Ersatzlösung, sodass die Graph-Export-Funktionalität vollständig implementiert werden konnte.

\subsection{Code Processor}

Probleme bereitete hier das zuverlässige Finden des Java Development Kit (JDK) auf dem Rechner des Benutzers.
Per Environment-Variable kann der JDK-Pfad zwar ermittelt werden, allerdings nur, falls die Variable passend gesetzt ist und nicht stattdessen auf den Pfad des Java Runtime Environment (JRE) gesetzt ist.
Des Weiteren ist es unzuverlässig, den Pfad anderweitig automatisiert zu finden, da dieser sowohl von der verwendeten Java-Version als auch vom Betriebssystem abhängt.

Daher wurde entschieden, den Benutzer nach dem ersten Programmstart nach seinem JDK-Pfad zu fragen und diesen in einer editierbaren Konfigurationsdatei abzuspeichern.
Dies stellt zwar eine kleine Unannehmlichkeit für den Benutzer dar, sichert dafür aber die Zuverlässigkeit des Programms.

\subsection{Effiziente Worklist und weitere Analysen}

Diese Funktionalitäten wurden nicht implementiert, da der Implementierungsaufwand der Analysen aus den Muss-Kriterien enorm war und keine zeitliche Kapazität übrig ließ.

Allerdings gilt auch hier, dass aufgrund der erweiterbaren Entwurfsarchitektur die erforderlichen Änderungen an existierenden Klassen minimal sind, falls nachträglich Analysen oder \inlinecode{Worklist}-Subklassen hinzugefügt werden sollen.

	\part{Tools}
	\part{Änderungen des Entwurfs}

\section{DFAFramework}

Die größte Änderung besteht in der Aufteilung des Interfaces \inlinecode{Lattice} in die drei Interfaces \inlinecode{Initializer}, \inlinecode{Transition} und \inlinecode{Join}, jeweils mit dem Typparameter \inlinecode{E extends LatticeElement}, der das verwendete \inlinecode{LatticeElement} angibt.
Dabei hat jedes der drei neuen Interfaces genau eine Methode, die die von dem Interface dargestellte Operation beschreibt:

\begin{itemize}
	\item \inlinecode{Initializer<E extends LatticeElement>}: initialisiert eine Analyse mittels \inlinecode{getInitialStates():Map<Block,BlockState<E>>}
	\item \inlinecode{Join<E extends LatticeElement>}: führt Join-Operationen mittels \inlinecode{join(Set<E>):E} aus
	\item \inlinecode{Transition<E extends LatticeElement>}: führt Transitionen mittels \inlinecode{transition(E,Unit):E} aus
\end{itemize}

Diese Aufteilung ermöglicht eine vollständig modulare Implementierung der Analysen.
\inlinecode{Initializer}, \inlinecode{Join} und \inlinecode{Transition} sind vollständig austauschbar (vorausgesetzt die typparameter sind kompatibel).

Die ehemals abstrakte Klasse \inlinecode{DataFlowAnalysis<E extends LatticeElement>} ist nun ein Interface, welches von den Interfaces \inlinecode{Initializer}, \inlinecode{Join} und \inlinecode{Transition} erbt.
Damit stellt \inlinecode{DataFlowAnalysis} alle für die Ausführung einer Datenflussanalyse benötigten Operationen bereit.

Zusätzlich zum Interface \inlinecode{DataFlowAnalysis} wurde noch die abstrakte Klasse \inlinecode{CompositeDataFlowAnalysis<E extends LatticeElement>} hinzugefügt, die \inlinecode{DataFlowAnalysis} implementiert und einen Konstruktor bereitstellt, der einen \inlinecode{Initializer}, einen \inlinecode{Join} und eine \inlinecode{Transition} entgegennimmt.
\inlinecode{getInitialStates}, \inlinecode{join} und \inlinecode{transition} werden dann an die entsprechenden Methoden der übergebenen Objekte delegiert.
Dies erlaubt das Zusammensetzten neuer Datenflussanalysen aus einzelnen Komponenten (vorausgesetzt die Typparameter sind kompatibel).

Die Rolle der ehemaligen \inlinecode{DataFlowAnalysis} übernimmt die abstrakte Klasse \inlinecode{DFAFactory<E extends LatticeElement>}.
Damit wurde der Rückgabetyp der Methode \inlinecode{getAnalysis(String)} von \inlinecode{AnalysisLoader} auch zu \inlinecode{DFAFactory} angepasst.

Weiter ist die Klasse \inlinecode{SimpleBlockGraph} hinzugekommen.
Diese erbt von \inlinecode{soot.toolkits.graph.BriefBlockGraph} und sorgt lediglich dafür, dass die \inlinecode{getTails}-Methode höchstens einen \inlinecode{Block} zurückgibt (dass es also höchstens einen Endblock gibt). Dies wird durch Einfügen eines künstlichen Endblocks erreicht, falls der ursprüngliche \inlinecode{BriefBlockGraph} mehrere Endblöcke hat.

Eine weitere neue Klasse ist \inlinecode{DFAPrecalcController}, welche die Vorberechnung einer \inlinecode{DFAExecution} kontrolliert. Zu diesem Zweck wurde dem Konstruktor ein Parameter vom Typ \inlinecode{DFAPrecalcController} hinzugefügt.
Beispielsweise kann die Vorberechnung mittels \inlinecode{pause(waitTime : int)} pausiert werden und dann mit \inlinecode{continue()} fortgesetzt werden. 
Dies ist z. B. nützlich, um zu verhindern, dass zu viel Speicher belegt wird, während sich der Nutzer längere Zeit in einem Menü befindet. 
Eine Vorberechnung kann auch mit \inlinecode{stop()} gestoppt werden.
Dann wird ein Zwischenergebnis gespeichert und die Vorberechnung abgebrochen.

Als Exception-Klassen wurden \inlinecode{DFAException} sowie die Unterklassen \inlinecode{UnsupportedValueException} und \inlinecode{UnsupportedStatementException} hinzugefügt.
Diese werden benutzt um Probleme bei der Ausführung einer Datenflussanalyse zu signalisieren (z. B. wenn nicht unterstützte Instruktionen angetroffen werden, die nicht einfach ignoriert werden können).

\subsection{Kleine Änderungen}

Im Folgenden werden weitere kleine Änderungen zusammengefasst:

\inlinecode{Color} wurde umbenannt in \inlinecode{LogicalColor}.

Der Parameter \inlinecode{cpy : AnalysisState} im Konstruktor \inlinecode{AnalysisState(..., cpy : AnalysisState)} wurde ersetzt durch die beiden Parameter \inlinecode{stateMap : Map<AbstractBlock, BlockState<E>>} und \inlinecode{colorMap : Map<BasicBlock, LogicalColor>}.
Weiter wurden die \inlinecode{(protected)} Methoden \inlinecode{setStateMap(Map<AbstractBlock, BlockState<E>>)}, \inlinecode{getStateMap():Map<AbstractBlock, BlockState<E>>}, \inlinecode{setColorMap(Map<BasicBlock, LogicalColor>)}, \inlinecode{getColorMap():Map<BasicBlock, LogicalColor>} zum setzen bzw. erhalten der \inlinecode{Map}s für die \inlinecode{BlockState}s und \inlinecode{LogicalColor}s hinzugefügt.
Die Methoden \inlinecode{getState(...):BlockState<E>} und \inlinecode{setState(...)} wurden zu \inlinecode{getBlockState(...):BlockState<E>} und \inlinecode{setBlockState(...)} umbenannt.

In der Klasse \inlinecode{BlockState<E extends LatticeElement>} wurden die beiden \inlinecode{(protected)} Methoden \inlinecode{setInState(E)} und \inlinecode{setOutState(E)} hinzugefügt.

In der Klasse \inlinecode{BasicBlock} wurde die \inlinecode{(protected)} Methode \inlinecode{setElementaryBlocks(List<ElementaryBlock>)} hinzugefügt.
Weiter wurden \inlinecode{setCorrespondingSootBlock(...)} und \inlinecode{getCorrespondingSootBlock()} zu \inlinecode{setSootBlock(...)} und \inlinecode{getSootBlock()} umbenannt.

In der Klasse \inlinecode{ElementaryBlock} wurde die \inlinecode{(protected)} Methode \inlinecode{setUnit(Unit)} hinzugefügt.

\section{GUI}

\subsection{Hinzugefügte Klassen}


Das GUI-Package hat 11 neue Klassen gegenüber dem Entwurf aufzuweisen. 10 davon sind reine \enquote{Utility}-Klassen und \inlinecode{Enum}-Klassen für Konstanten.

\begin{itemize}
	\item \textbf{Colors:}
Ein \inlinecode{Enum}, welches Konstanten für die Farben der Benutzeroberfläche beinhaltet.

	\item \textbf{ControlPanelState:} Ein \inlinecode{Enum}, welches die möglichen Zustände des \inlinecode{ControlPanel} beinhaltet.

	\item \textbf{Option:} Ein \inlinecode{Enum}, welches bei Dialogen mit verschiedenen Optionen die vom Benutzer ausgewählte Option repräsentiert.

	\item \textbf{Quality:} Ein \inlinecode{Enum}, welches die ausgewählte Qualität in der \inlinecode{GraphExportBox} repräsentiert.

	\item \textbf{IconLoader:} Eine \enquote{Utility}-Klasse, welche eine statische Methode zur Verfügung stellt, um Bilder für die Benutzeroberfläche in das Programm zu laden und zu skalieren.

	\item \textbf{JComponentDecorator:} Eine \enquote{Utility}-Klasse, welche auf \inlinecode{JComponent}s Standardwerte setzt, damit diese nicht für jede Komponente einzeln gesetzt werden müssen. Dies reduziert Codeverdopplung.

	\item \textbf{JButtonDecorator:} Eine \enquote{Utility}-Klasse, welche einen \inlinecode{JButton} zuerst dem \inlinecode{JComponentDecorator} übergibt und dann selbst Standardwerte für \inlinecode{JButton}s setzt.

	\item \textbf{JLabelDecorator:} Eine \enquote{Utility}-Klasse, welche ein \inlinecode{JLabel} zuerst dem \inlinecode{JComponentDecorator} übergibt und dann Standardwerte für \inlinecode{JLabel}s setzt.

	\item \textbf{JSliderDecorator:} Eine \enquote{Utility}-Klasse, welche einen \inlinecode{JSlider} zuerst dem \inlinecode{JComponentDecorator} übergibt und dann Standardwerte für \inlinecode{JSlider} setzt.

	\item \textbf{GridBagConstraintFactory:} Eine \enquote{Utility}-Klasse, welche Standard-\inlinecode{GridBagConstraints} für eine \inlinecode{Swing}-Komponente erstellt. Diese bestimmen Platz und Größe einer Komponente im \inlinecode{GridBagLayout}.

\end{itemize}

Außerdem hat eine Komponente mehr Funktionalität gebraucht, als im Entwurf veranschlagt war. Diese ist das \inlinecode{CodeField}.
Im Entwurf war eine einfache \inlinecode{JTextArea} vorgesehen, jetzt ist das \inlinecode{CodeField} eine Klasse die von \inlinecode{JScrollPane} erbt und zwei \inlinecode{JTextArea}s beinhaltet.

\subsection{Hinzugefügte Methoden}


Einige Klassen haben neue Funktionalität hinzugefügt bekommen.

\begin{itemize}

	\item \textbf{public File getCompilerPath():} In der \inlinecode{ProgramFrame}-Klasse. Der Benutzer muss beim ersten Starten des Programms den Pfad zu seiner \inlinecode{JDK} angeben, damit sein Programm-Code kompiliert werden kann. Dieser Pfad kann vom \inlinecode{Controller} hier abgefragt werden.

	\item \textbf{public void setCode(String code):} In der \inlinecode{InputPanel}-Klasse.
Übergibt einen \inlinecode{String} an das \inlinecode{CodeField}.

	\item \textbf{public void reset():} In der \inlinecode{StatePanelOpen}-Klasse. Setzt den Inhalt dieses \inlinecode{JPanel}s auf den Ausgangszustand zurück.

	\item \textbf{public Option getOption():} In der \inlinecode{MethodSelectionBox} und der \inlinecode{GraphExportBox}. Hierüber kann die ausgewählte Option des Benutzers abgefragt werden.

\end{itemize}

\subsection{Geänderte Klassen}


Zwei Klassen wurden umbenannt und eine dieser Klassen hat eine leicht andere Funktion.

\begin{itemize}
	\item \textbf{WarningBox:} wurde in \inlinecode{OptionBox} umbenannt und hat nun drei statt zwei \inlinecode{JButtons} zum Auswählen.

	\item \textbf{AlertBox:} wurde in \inlinecode{MessageBox} umbenannt.
\end{itemize}

\section{CodeProcessor}

\subsection{Geänderte Klassen}

\begin{itemize}
	\item \textbf{<<Interface>> Filter:} wurde zu \textbf{class Filter} geändert, da diese übergeordnete Klasse dafür sorgt, dass die Methoden, die manuell für die Taint-Analyse hinzugefügt wurden, herausgefiltert werden, egal welcher abgeleitete Filter verwendet wird.
	
	\item \textbf{NoFilter implements Filter} wurde zu \textbf{NoFilter extends Filter} geändert.
	
	\item \textbf{StandardFilter implements Filter} wurde zu \textbf{StandardFilter extends Filter} geändert.
\end{itemize}

\subsection{Hinzugefügte Methoden}

\begin{itemize}
	\item \textbf{public boolean filterTaint(SootMethod method):} In der \inlinecode{Filter}-Klasse. Diese Methode filtert die für die Taint-Analyse manuell hinzugefügten Methoden heraus, damit sie dem Benutzer nicht angezeigt werden.
\end{itemize}

\subsection{Geänderte Methoden}

\begin{itemize}
	\item \textbf{public String getPackageName():} in der \inlinecode{CodeProcessor}-Klasse wurde in \textbf{public String getPath():} umbenannt, da diese Methode den absoluten Pfad zu der gespeicherten .class-Datei zurück gibt.
	
	\item \textbf{public BlockGraph buildGraph(String methodSignature):} in der \inlinecode{GraphBuilder}-Klasse wurde zu \textbf{public SimpleBlockGraph buildGraph(String methodSignature):} geändert, da ein \inlinecode{SimpleBlockGraph} genauere Voraussetzungen spezifiziert.
	
	\item \textbf{public boolean filter(String methodSignature):} in der \inlinecode{Filter}-Klasse wurde zu \textbf{public boolean filter(SootMethod method):} geändert, da man mit Hilfe von Soot herausfinden kann, ob eine \inlinecode{SootMethod} eine Standard Java Methode ist oder nicht. Diese Methode wird von den abgeleiteten Klassen \inlinecode{NoFilter} und \inlinecode{StandardFilter} implementiert.
\end{itemize}

\subsection{Entfernte Methoden}

\begin{itemize}
	\item \textbf{public String getBinaries():} In der \inlinecode{CodeProcessor}-Klasse. Da diese Klasse die generierte .class-Datei lokal beim Benutzer speichert, wird diese Methode nicht mehr benötigt.
\end{itemize}


\section{Controller}

\subsection{Hinzugefügte Klasse}

\begin{itemize}
	\item \textbf{App:} In dieser Klasse befindet sich die main-Methode, welche beim Programmstart ausgeführt wird. Sie erstellt einen \inlinecode{Controller} und ein \inlinecode{ProgramFrame} und setzt für die Funktionalität des Programms wichtige Einstellungen.
\end{itemize}

\subsection{Hinzugefügte Methoden}

\begin{itemize}
	\item \textbf{public void completedAnalysis():} In der Klasse \inlinecode{Controller}. Diese Methode wird ausgeführt, nachdem die Vorberechnung Analyse beendet wurde. Sie setzt die \inlinecode{DFAExecution}, Einstellung am \inlinecode{ControlPanel} und startet den Aufbau eines CFG im \inlinecode{GraphUIController}.
	
	\item \textbf{public void createExceptionBox(String message):} In der Klasse \inlinecode{Controller}. Diese Methode wird vom \inlinecode{DFAPrecalculator} aufgerufen, wenn während der Berechnung der Analyse eine Exception auftritt. Nähere Informationen zu dieser Exception werden dem Bentuzer in einer \inlinecode{MessageBox} ausgegeben.
	
	\item \textbf{public int getDelay():} In der Klasse \inlinecode{Controller}. Diese Methode gibt das aktuell im \inlinecode{ControlPanel} ausgewählte Delay zurück. Damit kann dieser Wert auch während der automatischen Wiedergabe der Analyseschritte geändert werden.
	
	\item \textbf{public List<String> getWorklist():} In der Klasse \inlinecode{Controller}. Diese Methode gibt eine Liste alle verfügbarere \inlinecode{Worklist}s aus. Diese werden in der GUI dargestellt.
	
	\item \textbf{public int getDelay():} In der Klasse \inlinecode{Controller}. Diese Methode gibt das aktuell im \inlinecode{ControlPanel} ausgewählte Delay zurück. Damit kann dieser Wert auch während der automatischen Wiedergabe der Analyseschritte geändert werden.
	
	\item \textbf{public void pathSelection():} In der Klasse \inlinecode{Controller}. Diese Methode wird direkt bei Programmstart ausgeführt. Sie sorgt dafür, dass der Benutzer nach dem Pfad seiner JDK gefragt wird, falls dieser noch nicht in der Konfigurationsdatei gespeichert ist. 
	
	\item \textbf{public void setDefaultCode():} In der Klasse \inlinecode{Controller}. Diese Methode sorgt dafür, das ein Beispielcode im Editor bei Programmstart dargestellt wird, mit welchem eine erste Analyse durchgeführt werden kann.
	
	\item \textbf{public boolean shouldContinue():} In der Klasse \inlinecode{Controller}. Diese Methode wird während der automatischen Wiedergabe mehrmals aufgerufen. Sie bestimmt ob diese weiter ausgeführt oder abgebrochen wird, falls ein Breakpoint erreicht wurde oder der Benutzer auf Pause geklickt hat.
	
	\item \textbf{protected void visibilityInput(), protected void visibilityWorking(), protected void visiblityPlaying(), protected void visibilityPrecalculating():} In der Klasse \inlinecode{Controller.} Diese Methoden setzten die Sichtbarkeit und Modifizierbarkeit der unterschiedliche Panels der GUI, in Abhängigkeit vom aktuellen Zustand des Programms.
\end{itemize}

\subsection{Geänderte Methoden}

\begin{itemize}
	\item \textbf{public void DFAPrecalculator(DataFlowAnalysis dfa, Worklist w, Blockgraph bg):} In der \inlinecode{DFAPrecalculator}-Klasse wurde in \textbf{public void DFAPrecalculator(DFAFactory<? extends LatticeElement> dfa, Worklist w, SimpleBlockGraph simpleBg, DFAPrecalcController precalcController, Controller controller):} geändert. Diese Änderungen ergeben sich durch die Entwurfsänderung des Packages DFAFramework.
\end{itemize}


\subsection{Entfernte Methoden}

\begin{itemize}
	\item \textbf{public DFAExecution getDFAExecution():} In der \inlinecode{DFAPrecalculator}-Klasse. Durch die hinzugefügte Klasse \inlinecode{DFAPrecalcController}, mit deren Instanz eine DFAExecution erhalten werden kann, wird diese Methode nicht mehr benötigt.
\end{itemize}


\end{document}