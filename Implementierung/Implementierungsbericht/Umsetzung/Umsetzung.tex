%note: don't split this document up with include{...}

\part{Umsetzung}

In diesem Kapitel wird erklärt, welche Funktionalität aus dem Pflichtenheft implementiert wurde und in welchem Ausmaß der Implementierungsplan erfüllt wurde.
Insbesondere wird dabei auf die Behandlung von unerwarteten Problemen während der Implementierung eingegangen.

\section{Umsetzung der Kriterien}
% TODO: Update if needed, e.g. Reaching Definitions. (Slides as well)

\subsection{Muss-Kriterien}

Die Muss-Kriterien wurden planmäßig und vollständig implementiert. Diese sind:

\begin{itemize}
  \item Eingabe von Code und Verarbeitung zu einem CFG
  \item Unterstützung des definierten Java-Subsets
  \item Anzeige des CFG
  \item Ausführung und Anzeige der implementierten Analysen
  \item Automatische Fixpunktiteration
  \item Abbruch der Analyse 
\end{itemize}

Insbesondere wurden alle in den Muss-Kriterien geforderten Analysen implementiert.
Diese sind Constant Folding, Reaching Definitions, Constant Bits und Taint Analysis.

\subsection{Kann-Kriterien}
% TODO: Update if needed, e.g. Reaching Definitions. (Slides as well)

Die Kann-Kriterien wurden zum Großteil implementiert.
Von insgesamt 19 Kann-Kriterien wurden 15 vollständig implementiert, ein weiteres (Auswahlmöglichkeit der Worklistalgorithmen) teilweise.
Lediglich drei Kann-Kriterien wurden \textit{nicht} implementiert. Dazu gehören:

\begin{itemize}
  \item Zeilenangabe bei Fehlermeldungen
  \item Weitere Datenflussanalysen (Live Variables, Available Expressions, Definite Assignment)
  \item Manual Page
\end{itemize}

\newpage
\section{Problembehandlung}

Es traten einige unvorhersehbare Probleme auf, die eine Änderung der geplanten Implementierung erforderten.

\subsection{Dynamisches Laden von Klassen für eigene Datenflussanalysen}

Es war vorgesehen, zum dynamischen Laden von Klassen für eigene Datenflussanalysen des Benutzers die Reflections-Bibliothek zu verwenden.
Diese hatte in einem Prototyp einwandfrei funktioniert.
Im Laufe der Implementierung stellte sich jedoch heraus, dass die Reflections-Bibliothek im Zusammenspiel mit der für Datenflussanalysen verwendeten Soot-Bibliothek einen Abhängigkeitskonflikt verursacht.
Dies kommt dadurch zustande, dass die beiden Bibliotheken unterschiedliche Versionen der selben Abhängigkeit benötigen. 
Daraus resultiert ein Laufzeitfehler, der nicht trivial behebbar ist.

Wegen des großen erwarteten Aufwands wurde ausgeschlossen, eine der Bibliotheken anzupassen. 
Es wurde entschieden, andere (insbesondere Muss-) Kriterien zuerst zu implementieren und dynamisches Laden von Klassen nur bei ausreichender zeitlicher Kapazität selbst zu implementieren.
Zu diesem Zeitpunkt lassen sich daher im Programm keine Klassen dynamisch laden.

Allerdings muss hervorgehoben werden, dass die Entwurfsarchitektur des Programmes darauf ausgelegt ist, dass jederzeit neue Datenflussanalysen implementiert werden können und das Fehlen der dynamischen Funktionalität kein großes Hindernis darstellt.
Lediglich der letzte Schritt zur Implementierung eigener Analysen fehlt also – dies lässt sich durch Rekompilieren des Programms umgehen.
Darüber hinaus kann dynamisches Laden von Klassen nachgereicht werden, ohne nennenswerte Teile des Programms umschreiben zu müssen.

Daher wird das Kriterium \enquote{Implementierung einer eigenen Datenflussanalyse} als \textit{vollständig implementiert} betrachtet.

\newpage
\subsection{Graph-Export}

Bereits vor Beginn der Implementierungsphase wurde sichergestellt, dass eine benutzbare API für den Graph-Export in der JGraphX-Bibliothek zur Verfügung steht.
Um auch den Zustand der Analyse im Bild anzeigen zu können, sollte das gesamte \inlinecode{StatePanelOpen} dort hineingerendert werden.
Dies erwies sich allerdings als fehlerbehaftet und unzuverlässig. 
Insbesondere trat das Problem auf, dass einzelne Komponenten des Panels im Bild nicht angezeigt wurden.
Darüber hinaus ließ die Qualität des Font-Renderings derart zu wünschen übrig, dass entschieden wurde, das \inlinecode{StatePanelOpen} nicht für den Graph-Export zu verwenden.

Stattdessen wurde eine Ersatzimplementierung in das Programm eingebaut, welches den Text des Analyse-Zustands mit Hilfe der Swing-Klasse \inlinecode{Graphics2D} manuell in das Bild einfügt.
Dies verursachte zwar etwas Verzögerung, erwies sich jedoch als akzeptable Ersatzlösung, sodass die Graph-Export-Funktionalität vollständig implementiert werden konnte.

\subsection{Code Processor}

Probleme bereitete hier das zuverlässige Finden des Java Development Kit (JDK) auf dem Rechner des Benutzers.
Per Environment-Variable kann der JDK-Pfad zwar ermittelt werden, allerdings nur, falls die Variable passend gesetzt ist und nicht stattdessen auf den Pfad des Java Runtime Environment (JRE) gesetzt ist.
Des Weiteren ist es unzuverlässig, den Pfad anderweitig automatisiert zu finden, da dieser sowohl von der verwendeten Java-Version als auch vom Betriebssystem abhängt.

Daher wurde entschieden, den Benutzer nach dem ersten Programmstart nach seinem JDK-Pfad zu fragen und diesen in einer editierbaren Konfigurationsdatei abzuspeichern.
Dies stellt zwar eine kleine Unannehmlichkeit für den Benutzer dar, sichert dafür aber die Zuverlässigkeit des Programms.

\subsection{Effiziente Worklist und weitere Analysen}

Diese Funktionalitäten wurden nicht implementiert, da der Implementierungsaufwand der Analysen aus den Muss-Kriterien enorm war und keine zeitliche Kapazität übrig ließ.

Allerdings gilt auch hier, dass aufgrund der erweiterbaren Entwurfsarchitektur die erforderlichen Änderungen an existierenden Klassen minimal sind, falls nachträglich Analysen oder \inlinecode{Worklist}-Subklassen hinzugefügt werden sollen.
