%note: don't split this document up with include{...}

\section{Funktionale Anforderungen}

\functionality{Zeilennummerierung}{fnc:LineNumbers}
Im Textfeld für die Eingabe des Codes gibt es eine Zeilennummerierung.

\functionality{Code-Eingabe mit Editor}{fnc:CodeInputIntern}
\fulfills{crt:CodeInput}
Es gibt ein Textfeld, in welches der Benutzer Code eingeben kann, welcher anschließend zu einem CFG verarbeitet werden kann.

\functionality{Datei-Eingabe}{fnc:CodeInputExt}
\fulfills{crt:CodeInputExto}
Der Benutzer kann einzulesenden Code aus einer .java-Datei laden.

\functionality{Java-Subset}{fnc:minimalSubset}
\fulfills{crt:minSubset}
Das Programm unterstützt ein minimales Subset von Java.
Unterstützt werden:
\begin{enumerate}[label=(\alph*)]
\item Die arithmetischen Operationen +, -, *, /,
\item die Kontrollflussstrukturen if, else, while,
\item die Vergleichsoperatoren <, >, ==,
\item die Datentypen int und boolean,
\item die Zuweisung =.
\end{enumerate}

\functionality{Erweitertes Subset}{fnc:extendedSubset1}
\fulfills{crt:extendedSubset1o}
Das Programm unterstützt zusätzlich zum minimalen Subset folgende Operatoren:
\begin{enumerate}[label=(\alph*)]
\item Den Modulo-Operator \%,
\item die Bitshift-Operatoren <\null< und >\null>,
\item die bitweisen Operatoren \& und |.
\end{enumerate}

\newpage

\functionality{Syntaxfehler}{fnc:syntaxError}
\fulfills{crt:minSubset}
\fulfills{crt:lineNmbrErrorO}
Enthält der vom Benutzer eingegebene Code Syntaxfehler, so erhält der Benutzer eine Fehlermeldung, deren Informationsgehalt der Fehlermeldung des Compilers entspricht.

\functionality{Nichtunterstützte Operationen}{fnc:OpsError}
\fulfills{crt:minSubset}
\fulfills{crt:lineNmbrErrorO}
Der Benutzer erhält eine Fehlermeldung, falls in der zu analysierenden Funktion Operationen oder Kontrollflusskonstrukte vorhanden sind, welche nicht dem definierten Subset von Java entsprechen. Die Fehlermeldung weist auf das unterstützte Java-Subset hin.

\functionality{Anzeigen des CFG}{fnc:showCFGf}
\fulfills{crt:CFG}
Die GUI kann den aus dem Java-Code erstellten CFG in einem Zwischencode anzeigen. Der CFG ist in einem Panel (im Folgenden als Graph-Panel bezeichnet) dargestellt.

\functionality{Interaktion mit dem CFG 1}{fnc:interactCFG1f}
\fulfills{crt:GraphLayoutO}
\fulfills{crt:BreakpointsO}
Der Benutzer kann mit der GUI des CFG interagieren, um Breakpoints in einzelnen Zeilen zu setzen.

\functionality{Interaktion mit dem CFG 2}{fnc:interactCFG2f}
\fulfills{crt:GraphLayoutO}
Der Benutzer kann mit der GUI des CFG interagieren, um zu scrollen.

\functionality{Interaktion mit dem CFG 3}{fnc:interactCFG2bf}
\fulfills{crt:GraphLayoutO}
Der Benutzer kann mit der GUI des CFG interagieren, um zu zoomen.

\functionality{Interaktion mit dem CFG 4}{fnc:interactCFG3f}
\fulfills{crt:GraphLayoutO}
Der Benutzer kann mit der GUI des CFG interagieren, um Blöcke, beziehungsweise Zeilen auszuwählen.

\functionality{Auswahl der Datenflussanalysen}{fnc:pickAnalysis}
\fulfills{crt:DFA}
\fulfills{crt:DFAArtenO}
Der Benutzer kann zwischen allen implementierten Datenflussanalysen wählen und sie auf den CFG anwenden.

\newpage

\functionality{Blockweises Durchlaufen der Analyse}{fnc:blockAnalysis}
\fulfills{crt:ShowAnalysis}
\fulfills{crt:backwardsAnalysisO}
Der Benutzer kann die Analyse blockweise durchlaufen. Dafür stehen ihm folgende Buttons zur Verfügung:\par
\faFastForward\ : Springt einen Block weiter.\par
\faFastBackward\ : Springt einen Block zurück.

\functionality{Zeilenweises Durchlaufen der Analyse}{fnc:lineAnalysis}
\fulfills{crt:lineAnalysisO}
\fulfills{crt:backwardsAnalysisO}
Der Benutzer kann die Analyse zeilenweise durchlaufen. Dafür stehen ihm folgende Buttons zur Verfügung:\par
\faStepForward\ : springt eine Zeile weiter.\par
\faStepBackward\ : springt eine Zeile zurück.

\functionality{Abbruch der Analyse}{fnc:stopAnalysisf}
\fulfills{crt:AbortAnalysis}
Der Benutzer kann eine laufende Analyse abbrechen, um neuen Code analysieren zu lassen.

\functionality{Automatisches Durchlaufen der Analyse}{fnc:automaticAnalysis}
\fulfills{crt:Fixpunkt}
Der Benutzer kann den Button  \faPlay\  betätigen, um die Analyse automatisch bis zum Fixpunkt durchlaufen zu lassen.

\functionality{Automatisches Durchlaufen bis Breakpoint}{fnc:breakAnalysisf}
\fulfills{crt:BreakpointsO}
Der Benutzer kann den Button  \faPlay\  betätigen, um die Analyse automatisch bis zum nächsten Breakpoint durchlaufen zu lassen.

\functionality{Worklistalgorithmen}{fnc:worklistNaive}
\fulfills{crt:Fixpunkt}
Das Programm kann zur Fixpunktanalyse den naiven Worklistalgorithmus anwenden.

\functionality{Erweiterte Worklistalgorithmen}{fnc:moreWorklist}
\fulfills{crt:WorklistO}
Der Benutzer kann zwischen verschiedenen Worklistalgorithmen wählen.
Diese entscheiden, in welcher Reihenfolge die Blöcke in der Datenflussanalyse behandelt werden.
Es stehen folgende Algorithmen zur Auswahl:
\begin{enumerate}[label=(\alph*)]
\item \textbf{Naiv} \par
Es werden alle Blöcke in der Reihenfolge, in der sie „entdeckt“ werden (Breitensuche), in die Worklist geschrieben und anschließend in dieser Reihenfolge abgearbeitet.
\item \textbf{Random} \par
Die Blöcke werden wie beim naiven Worklistalgorithmus auf die Worklist geschrieben. Nach Abarbeitung eines Blocks in der Analyse wird zufällig ein Block aus der Worklist gewählt und als nächstes behandelt.
\item \textbf{Efficient} \par
Die Reihenfolge der Abarbeitung der Blöcke wird optimiert, sodass der Fixpunkt im Mittel mit einer geringeren Anzahl von Schritten gefunden wird.
\end{enumerate}

\functionality{Code-Speicherung}{fnc:saveCode}
\fulfills{crt:CodeInputExto}
Der Benutzer kann den Inhalt des Textfeldes zur Codeeingabe in einer .java-Datei abspeichern.

\functionality{Hotkeys}{fnc:Hotkeys}
\fulfills{crt:DFAo}
Dem Benutzer stehen Hotkeys zur Verfügung, mithilfe derer er die Datenflussanalyse steuern kann.

\functionality{Optisches Hervorheben der Blöcke}{fnc:Highlights}
\fulfills{crt:HighlightingO}
Je nach Status werden die Blöcke im CFG farbig hervorgehoben. Folgende Zustände werden unterschieden:
\begin{enumerate}[label=(\alph*)]
\item Noch nicht besucht.
\item Auf der Worklist.
\item Gerade ausgewählt.
\item Bereits besucht und nicht auf der Worklist.
\end{enumerate}

\functionality{Eingangs- und Ausgangszustand von Blöcken}{fnc:inOutBlockf}
\fulfills{crt:ShowAnalysis}
Dem Benutzer werden Eingangs- und Ausgangszustand des ausgewählten Blocks in einem GUI-Element angezeigt.

\functionality{Eingangs- und Ausgangszustand von Zeilen}{fnc:inOutLinef}
\fulfills{crt:lineAnalysisO}
Dem Benutzer werden Eingangs- und Ausgangszustand der ausgewählten Zeile in einem GUI-Element angezeigt.

\functionality{CFG-Export}{fnc:exportAsPicf}
\fulfills{crt:exportPicO}
Der Benutzer kann von allen Schritten der Analyse ein Bild des CFG abspeichern.

\functionality{Delay-Optionen}{fnc:Delayf}
\fulfills{crt:DelayO}
Der Benutzer kann zwischen den automatischen Schritten eine Zeitverzögerung einstellen. Diese reicht von 0 Sekunden künstlichem Delay (schnellstmögliches Durchlaufen) bis hin zu einigen Sekunden Verzögerung.

\functionality{Ungespeicherte Änderungen}{fnc:unsavedChangesf}
Gibt es ungespeicherte Änderungen im Editor, wenn der Benutzer das Programm schließen möchte, so wird er gefragt, ob er diese speichern möchte.
Falls der Benutzer eine neue .java-Datei öffnen möchte und ungespeicherte Änderungen vorliegen, so wird der Benutzer gewarnt, dass diese Änderungen verfallen.

\functionality{Mehrere Funktionen}{fnc:moreFncf}
\fulfills{crt:moreFncO}
Der Benutzer kann Java-Code einlesen oder eingeben, welcher mehrere Funktionen
enthält. Liegen mehrere Funktionen vor, so fragt das Programm vor Start der Analyse, welche Funktion analysiert werden soll.

\functionality{Jump to Action}{fnc:jumpToActionf}
\fulfills{crt:JumpToActionO}
Der Benutzer kann mithilfe eines Buttons das automatische Springen zum aktiven Block, beziehungsweise zur aktiven Zeile aktivieren. Der Button bleibt aktiv, bis er erneut betätigt wird.

\functionality{Manual Page}{fnc:ManPagef}
\fulfills{crt:ManPageO}
Dem Programm liegt eine Manual Page bei, in welcher der Benutzer über die wichtigsten Funktionen des Programms informiert wird.

\newpage

\functionality{Fortschrittsanzeige}{fnc:ProgressBarf}
\fulfills{crt:ProgressBarO}
Das Programm besitzt einen Fortschrittsbalken. Dieser zeigt zu jedem Zeitpunkt der Analyse, wie viele Schritte im Verhältnis zur Gesamtzahl der Schritte bereits erfolgt sind.

\functionality{Springen zu beliebigem Analysepunkt}{fnc:Sliderf}
\fulfills{crt:SliderO}
Das Programm besitzt einen Slider, mit welchem der Benutzer zu bestimmten Punkten der Analyse springen kann. Hierbei wird immer an den Beginn eines Blocks gesprungen.

\functionality{Implementieren einer eigenen Datenflussanalyse}{fnc:OwnAnalysef}
\fulfills{crt:OwnAnalyseO}
Der Benutzer kann eigene Datenflussanalysen implementieren. Dazu kann er Java-Plugins schreiben, die dynamisch geladen werden. Es steht eine API zur Verfügung, mittels derer Datenflussanalysen implementiert werden können.