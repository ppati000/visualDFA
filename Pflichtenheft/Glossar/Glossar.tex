\newglossaryentry{CFG}{
	name=Kontrollflussgraph (CFG),
	description={
	Ein Graph, dessen Knoten die Grundblöcke eines Programms oder einer Methode sind. Die Kanten des CFG geben an, wie der Kontrollfluss zwischen den Grundblöcken wechseln kann.
	}
}

\newglossaryentry{Block}{
	name=Grundblock,
	description={
	Ein Knoten des Kontrollflussgraphen.
	}
}

%todo umformulieren, präziser
\newglossaryentry{DFA}{
	name=Datenflussanalyse,
	description={
	Eine statische Analyse, die den CFG auf bestimmte Merkmale (zum Beispiel Optimierbarkeit) untersucht.
	}
}

\newglossaryentry{InOutState}{
	name=Eingangs- und Ausgangszustand,
	description={
	Jeder Block des Kontrollflussgraphen hat einen Eingangszustand und einen Ausgangszustand.
	Diese entsprechen den Informationen, die vor bzw. nach Durchlaufen des Blocks bekannt sind.
	}
}

\newglossaryentry{UeFkt}{
	name=Übergangsfunktion,
	description={
	Eine Übergangsfunktion für einen Block bildet dessen Eingangszustand auf seinen Ausgangszustand ab. 
	Sie unterscheidet sich je nach Datenflussanalyse und Code des zugehörigen Blocks.
	}
}

\newglossaryentry{FP}{
	name=Fixpunkt,
	description={
	Der Fixpunkt einer Datenflussanalyse ist der Punkt, an welchem sich bei keiner Übergangsfunktion im gesamten Kontrollflussgraphen eine Veränderung des Ausgangszustandes des jeweiligen Knotens ergibt.
	}
}

\newglossaryentry{Java}{
	name=Java,
	description={
	Eine weit verbreitete Programmiersprache, welche in diesem Programm dazu benutzt werden kann, den zu analysierenden Code einzugeben.
	}
}

\newglossaryentry{GUI}{
	name=GUI,
	description={
	Graphical User Interface oder Graphische Benutzeroberfläche.
	Alle Programmelemente, die für den Benutzer sichtbar sind und Interaktion mit dem Programm ermöglichen.
	}
}

\newglossaryentry{GraphPanel}{
	name=Graph-Panel,
	description={
	Das GUI-Element, in welchem dem Benutzer der CFG angezeigt wird.
	}
}

\newglossaryentry{Worklist}{
	name=Worklist,
	description={
	Eine Datenstruktur, die die noch zu bearbeitenden Grundblöcke während einer Datenflussanalyse enthält und die Abarbeitungsreihenfolge festlegt.
	}
}

\newglossaryentry{WLAlgo}{
	name=Worklistalgorithmus,
	description={
	Ein Algorithmus, welcher eine Datenflussanalyse unter Verwendung einer Worklist ausführt.
	Nach Abarbeitung eines Blockes werden alle Blöcke, deren Eingangszustand sich dadurch geändert hat, zur Worklist hinzugefügt.
	}
}

\newglossaryentry{naiv}{
	name=Naiver Worklistalgorithmus,
	description={
	Es werden alle Knoten in der Reihenfolge, in der sie \enquote{entdeckt} werden (Breitensuche),
	in die Worklist eingefügt und anschließend in dieser Reihenfolge abgearbeitet.
	}
}

\newglossaryentry{random}{
	name=Zufälliger Worklistalgorithmus,
	description={
	Die Blöcke werden wie bei dem naiven Worklistalgorithmus in die Worklist eingefügt.
	Nach Abarbeitung jedes Knotens in der Analyse wird ein zufälliger neuer dieser Blöcke gewählt und behandelt.
	}
}

\newglossaryentry{efficient}{
	name=Effizienter Worklistalgorithmus,
	description={
	Die Reihenfolge der Abarbeitung der Blöcke wird optimiert, sodass der Fixpunkt mit möglichst wenig Iterationsschritten gefunden wird.
	}
}