\newglossaryentry{CFG}{
	name=Kontrollflussgraph (CFG),
	description={Ein Graph bestehend aus Knoten und Kanten. Die Knoten bestehen aus zusammenhängenden Code-Zeilen. Gibt es Kontrollflussstrukturen im Code, so werden diese durch durch Kanten zwischen den Blöcken dargestellt.}
}

\newglossaryentry{Blocks}{
	name=Grundblock,
	description={Ein Knoten des Kontrollflussgraphen}
}

\newglossaryentry{Datenflussanalyse}{
	name=Datenflussanalyse,
	description={Eine Analyse, um den CFG auf bestimmte 			Merkmale (zum Beispiel Optimierbarkeit) zu untersuchen.}
}

\newglossaryentry{InOutState}{
	name=Eingangs- und Ausgangszustand,
	description={Jeder Block des Kontrollflussgraphen hat einen Eingangszustand und einen Ausgangszustand. Diese enthalten die Datenwerte vor und nach Durchlaufen des Blocks.}
}

\newglossaryentry{Uebergangsfunktion}{
	name=Übergangsfunktion,
	description={Eine Übergangsfunktion für einen Block bildet dessen Eingangszustand auf seinen Ausgangszustand ab. Sie ist je nach Datenflussanalyse und Code im jeweiligen Block unterschiedlich.}
}

\newglossaryentry{Fixpunkt}{
	name=Fixpunkt,
	description={Der Fixpunkt einer Datenflussanalyse ist der Punkt, an welchem sich bei keiner Übergangsfunktion im gesamten Kontrollflussgraphen eine Veränderung des Ausgangszustandes des jeweiligen Knotens ergibt.}
}

\newglossaryentry{Java}{
	name=Java,
	description={Eine weit verbreitete Programmiersprache, welche in diesem Programm dazu benutzt werden kann, den zu analysierenden Code in das Programm einzugeben.}
}

\newglossaryentry{GUI}{
	name=GUI,
	description={Graphical User Interface oder Graphische Benutzeroberfläche. Alle Programmelemente, die für den Benutzer sichtbar sind und Interaktion mit dem Programm ermöglichen.}
}

\newglossaryentry{GraphPanel}{
	name=Graph-Panel,
	description={Das GUI-Element, in welchem der CFG dem Benutzer angezeigt wird.}
}

\newglossaryentry{Worklist}{
	name=Worklist,
	description={Die Worklist beinhaltet die Reihenfolge, in welcher die Blöcke des Kontrollflussgraphen in der Analyse abgearbeitet werden.}
}

\newglossaryentry{Worklistalgorithmus}{
	name=Worklistalgorithmus,
	description={Ein Algorithmus, welcher eine Datenflussanalyse unter Verwendung einer Worklist ausführt.}
}

\newglossaryentry{naiv}{
	name=Naiver Worklistalgorithmus,
	description={Es werden alle Knoten in der Reihenfolge in der sie \enquote{entdeckt} werden (Breitensuche)
in die Worklist geschrieben und anschließend in dieser Reihenfolge abgearbeitet.}
}

\newglossaryentry{random}{
	name=Zufälliger Worklistalgorithmus,
	description={Die Blöcke werden wie bei dem naiven Worklistalgorithmus auf die Worklist geschrieben. Am Ende der Abarbeitung jedes Knotens in der Analyse wird ein zufälliger neuer dieser Blöcke gewählt und behandelt.}
}

\newglossaryentry{efficient}{
	name=Effizienter Worklistalgorithmus,
	description={Die Reihenfolge der Abarbeitung der Blöcke wird optimiert, indem zuerst innere Schleifen durchlaufen werden, bis ein lokaler Fixpunkt gefunden wurde, bevor die äußere Schleife weiter analysiert wird.
}
}