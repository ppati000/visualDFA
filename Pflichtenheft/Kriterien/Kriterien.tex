%note: don't split this document up with include{...}

\section{Kriterien}

\subsection{Muss}

\criterium{Eingabe von Code}{crt:CodeInput}
Eingegebener Java-Code kann zu einem Kontrollflussgraphen (CFG) verarbeitet werden.

\criterium{Java-Subset}{crt:minSubset}
Das Programm benutzt eine minimale Teilmenge an Java-Operatoren, Kontrollflussstrukturen und Datentypen der Java Version 1.7. Bei Fehlern wird eine Fehlermeldung ausgegeben.

\criterium{Kontrollflussgraph}{crt:CFG}
Generierter Zwischencode wird dem Benutzer im Kontrollflussgraphen (CFG) angezeigt.

\criterium{Datenflussanalysen}{crt:DFA}
Das Programm beherrscht mehrere Datenflussanalysen, mit welchen es den Kontrollflussgraphen analysieren kann.
Diese sind:\par
\begin{enumerate}[label=(\alph*)]
\item Constant Folding
\item Reaching Definitions
\item Constant Bits
\item Taint Analysis.
\end{enumerate}

\criterium{Anzeigen der Analyse}{crt:ShowAnalysis}
Die Datenflussanalysen können dem Benutzer schrittweise animiert im Kontrollflussgraph angezeigt werden.

\criterium{Fixpunkt}{crt:Fixpunkt}
Das Programm kann den Fixpunkt der Analyse automatisch berechnen.

\criterium{Abbruch der Analyse}{crt:AbortAnalysis}
Der Benutzer kann die Analyse abbrechen und zur Code-Eingabe zurückkehren.

\subsection{Kann}

\criteriumOptional{Erweitertes Java-Subset 1}{crt:extendedSubset1o}
Das Programm kann Java-Code einlesen und verarbeiten, wenn dieser folgende Operatoren und Konstrukte beinhaltet:

Modulo Operator: \%

Bit-shift Operatoren: <<, >>

Logisches \glqq Und\grqq und \glqq Oder\grqq : \&, |

\criteriumOptional{Erweitertes Java-Subset 2}{crt:extendedSubset2o}
Das Programm kann Java-Code einlesen und verarbeiten, wenn dieser folgende Operatoren und Konstrukte beinhaltet:

for - Schleife (nicht for each)

else if - Bedingung

\criteriumOptional{Erweitertes Java-Subset 3}{crt:extendedSubset3o}
Das Programm kann Java-Code einlesen und verarbeiten, wenn dieser folgende Operatoren und Konstrukte beinhaltet:

Zuweisungen: +=, -=, *=, /=, \% =

\criteriumOptional{Zeilenangabe bei Fehlermeldungen}{crt:lineNmbrErrorO}
Entstehen beim Kompilieren des eingelesenen Codes Fehlermeldungen, so wird für den Nutzer ersichtlich, in welcher Zeile seines Java-Codes er den Fehler gemacht hat.

\criteriumOptional{Breakpoints in Datenflussanalyse}{crt:BreakpointsO}
Der Nutzer kann in seiner Datenflussanalyse Breakpoints setzen. Beim automatisierten Durchlaufen der Analyse stoppt diese am nächsten gesetzten Breakpoint.

\criteriumOptional{Analyse einzelner Zeilen innerhalb der Knoten}{crt:lineAnalysisO}
Beim Durchlaufen der Datenflussanalyse erhält der Nutzer nicht nur In- und Out-
Informationen zu den Knoten als Ganzes sondern auch zu den einzelnen Zeilen
innerhalb der Knoten.

\criteriumOptional{Dynamisches Graph-Layout}{crt:GraphLayoutO}
Der Nutzer hat die Möglichkeit das zunächst automatisierte Layout des
Kontrollflussgraphen selbst zu verändern, in dem er in den Graphen rein- und
rauszoomt und in dem er die einzelnen Knoten hin und her beweget.

\criteriumOptional{Einlesen von Code aus .java Datei}{crt:CodeInputExto}
Der Nutzer hat die Möglichkeit, eine .java Datei in das Programm zu laden, welche
den zu analysierenden Code enthält. Den eingelesenen Code kann er anschließend
auch im Programm bearbeiten.

\criteriumOptional{Delay-Optionen zwischen Autoplay-Schritten}{crt:DelayO}
Der Nutzer kann einstellen, wie lange die einzelnen Schritte im automatisierten
Durchlaufen der Datenflussanalyse angezeigt werden sollen. Hierbei reicht der
Optionsumfang von 0 Sekunden Delay (so schnell wie möglich zum Fixpunkt) bis hin
zu einigen Sekunden Anzeigezeit pro Schritt.

\criteriumOptional{Rückwärtsschritte in der Datenflussanalyse}{crt:backwardsAnalysisO}
Der Nutzer hat die Möglichkeit, in der Datenflussanalyse Schritte zurück zu gehen
und sich einen vorherigen Analysezustand erneut anzeigen zu lassen.

\criteriumOptional{Auswahlmöglichkeit der Worklistalgorithmen}{crt:WorklistO}
Der Nutzer kann zwischen verschiedenen Worklistalgorithmen wählen. Diese
entscheiden, in welcher Reihenfolge die Knoten in der Datenflussanalyse behandelt
werden. Es stehen folgende Algorithmen zur Auswahl:
\begin{enumerate}[label=(\alph*)]
\item \textbf{Naiv} \par
Von oben nach unten, von links nach rechts werden all Knoten in der Reihenfolge in
die Worklist geschrieben, in welcher sie \glqq entdeckt\grqq werden und dann in dieser Reihenfolge abgearbeitet.
\item \textbf{Random} \par
Die Blöcke werden wie beim naiven Worklistalgorithmus auf die Worklist geschrieben. Am Ende der Abarbeitung jedes Knotens in der Analyse wird ein
zufälliger neuer dieser Blöcke gewählt und behandelt.
\item \textbf{Efficient} \par
Die Reihenfolge der Abarbeitung der Blöcke wird optimiert, indem beispielsweise
zuerst innere Schleifen durchlaufen werden, bis ein lokaler Fixpunkt gefunden wurde, bevor die äußere Schleife weiter analysiert wird.
\end{enumerate}

\criteriumOptional{Speichern in .java Datei}{crt:saveO}
Der Nutzer kann den eingelesenen oder eingetippten Java-Code als .java Datei
speichern.

\criteriumOptional{.java File mit mehreren Funktionen}{crt:moreFncO}
Der Nutzer kann Java-Code einlesen oder eingeben, welcher mehrere Funktionen
enthält. Vor der Datenflussanalyse kann er dann auswählen, welche Methode
analysiert werden soll.

\criteriumOptional{\glqq Jump to action\grqq -Funktion}{crt:JumpToActionO}
Der Nutzer hat die Möglichkeit mit nur einem Mausklick den aktuell analysierten
Knoten beziehungsweise die aktuell analysierte Zeile als seinen ausgewählten Knoten beziehungsweise seine ausgewählte Zeile zu setzen, egal welchen anderen Knoten und in welcher Zeile im Graphen er sich zuvor befunden hat. Dabei wird auch die Ansicht automatisch auf den fraglichen Knoten ausgerichtet.

\criteriumOptional{Hotkeys für Datenflussanalyse}{crt:DFAo}
Dem Nutzer steht eine Reihe von Hotkeys zur Navigation durch die
Datenflussanalyse zur Verfügung.

\criteriumOptional{Weitere Datenflussanalysearten}{crt:DFAArtenO}
Der Nutzer kann zur Analyse seines eingegebenen oder eingelesenen Java-Codes
zusätzlich zu den Analysearten aus M7 aus 3 weiteren Arten der Datenflussanalyse
auswählen:
\begin{enumerate}[label=(\alph*)]
\item  Live Variable Analysis
\item Definite Assignment Analysis
\item Available Expression Analysis
\end{enumerate}

\criteriumOptional{Färbung des Graphen}{crt:HighlightingO}
Die Knoten des angezeigten Kontrollflussgraphen im Graph-Panel werden in
Abhängigkeit des aktuellen Analysestandes in 4 verschiedenen Farben eingefärbt:
\begin{enumerate}[label=(\alph*)]
\item Unvisited
\item On Worklist
\item In Action
\item Visited and not on Worklist
\end{enumerate}

\criteriumOptional{Exportieren des Graphen als Bild-Datei}{crt:exportPicO}
Der Nutzer hat die Möglichkeit den aktuell im Graph-Panel angezeigten Kontrollflussgraphen als Bilddatei zu exportieren. Er kann auch automatisiert mehrere Analyseschritte der Datenflussanalyse als Batch-Export exportieren.

\criteriumOptional{\glqq Speichern?\grqq -Meldung vor schließen}{crt:unsavedChangesO}
Der Nutzer wird gefragt, ob ungespeicherte Änderungen des .java files im Textfeld
speichern möchte, wenn er das Programm schließt und ungespeicherte Änderungen vorliegen.

\criteriumOptional{\glqq Ungespeicherte Änderungen verfallen\grqq -Meldung bei Laden einer neuen Datei}{crt:loadUnsavedO}
Der Nutzer bekommt eine Meldung, dass ungespeicherte Änderungen verfallen,
wenn eine neue .java Datei im Textfeld geladen werden soll und ungespeicherte
Änderungen vorliegen.

\criteriumOptional{Manual Page}{crt:ManPageO}
Dem Programm liegt eine Manual Page bei, in welcher der Benutzer über die wichtigsten Funktionen des Programms informiert wird.

\criteriumOptional{Fortschrittsanzeige}{crt:ProgressBarO}
Der Benutzer kann ablesen wie weit die Datenflussanalyse fortgeschritten ist.

\criteriumOptional{Springen zu beliebigem Analysepunkt}{crt:SliderO}
Der Benutzer kann an einen beliebigen Punkt der Datenflussanalyse springen.