%note: don't split this document up with include{...}

\section{Kriterien}

\subsection{Muss}

\criterium{Eingabe von Code}{crt:CodeInput}
Eingegebener Java-Code kann zu einem CFG verarbeitet werden.

\criterium{Java-Subset}{crt:minSubset}
Der zu analysierende Code benutzt eine minimale Teilmenge an Java-Operatoren, Kontrollflussstrukturen und Datentypen der Java-Version 1.7. Bei Fehlern wird eine Fehlermeldung ausgegeben.

\criterium{CFG}{crt:CFG}
Generierter Zwischencode wird dem Benutzer in einem CFG angezeigt.

\criterium{Datenflussanalysen}{crt:DFA}
Das Programm kann mehrere Datenflussanalysen ausführen und anhand des CFG visualisieren.
Diese sind:\par
\begin{enumerate}[label=(\alph*)]
\item Constant Folding
\item Reaching Definitions
\item Constant Bits
\item Taint Analysis
\end{enumerate}

\criterium{Anzeigen der Analyse}{crt:ShowAnalysis}
Die Schritte der Datenflussanalysen können dem Benutzer blockweise animiert im CFG angezeigt werden.

\criterium{Fixpunkt}{crt:Fixpunkt}
Das Programm kann den Fixpunkt der Analyse automatisch berechnen.

\newpage

\criterium{Abbruch der Analyse}{crt:AbortAnalysis}
Der Benutzer kann die Analyse abbrechen und zur Code-Eingabe zurückkehren.

\subsection{Kann}

\criteriumOptional{Erweitertes Java-Subset -Priorität 1 (hoch)}{crt:extendedSubset1o}
Das Programm kann Java-Code einlesen und verarbeiten, welcher weitere Operatoren und Konstrukte beinhaltet.

\criteriumOptional{Zeilenangabe bei Fehlermeldungen - Priorität 2 (mittel)}{crt:lineNmbrErrorO}
Entstehen beim Kompilieren des eingelesenen Codes Fehlermeldungen, so wird dem Benutzer angezeigt, in welcher Zeile des Java-Codes der Fehler gemacht wurde.

\criteriumOptional{Breakpoints in Datenflussanalyse - Priorität 2}{crt:BreakpointsO}
Der Benutzer kann Breakpoints in den erstellten CFG setzen. Beim automatisierten Durchlaufen der Analyse stoppt diese am nächsten gesetzten Breakpoint.

\criteriumOptional{Analyse einzelner Zeilen innerhalb der Blöcke - Priorität 1}{crt:lineAnalysisO}
Beim Durchlaufen der Datenflussanalyse erhält der Benutzer zusätzlich zu den In- und Out-Informationen der Blöcke auch In- und Out-Informationen zu den einzelnen Zeilen innerhalb der Blöcke.

\criteriumOptional{Dynamisches Graph-Layout - Priorität 1}{crt:GraphLayoutO}
Der Benutzer hat die Möglichkeit, das zunächst automatisiert erstellte Layout des CFG selbst zu verändern, indem er zoomen und Blöcke bewegen kann.

\criteriumOptional{Einlesen von Code aus .java-Datei und Speichern von Code in .java-Datei - Priorität 3 (niedrig)}{crt:CodeInputExto}
Der Benutzer hat die Möglichkeit, eine .java-Datei in das Programm zu laden, welche
den zu analysierenden Code enthält. Außerdem kann er im Editor des Programms geschriebenen Code in einer .java-Datei speichern.

\newpage

\criteriumOptional{Delay-Optionen zwischen Autoplay-Schritten - Priorität 1}{crt:DelayO}
Der Benutzer kann einstellen, wie lange die einzelnen Schritte im automatisierten
Durchlaufen der Datenflussanalyse angezeigt werden sollen.

\criteriumOptional{Rückwärtsschritte in der Datenflussanalyse - Priorität 1}{crt:backwardsAnalysisO}
Der Benutzer hat die Möglichkeit, in der Datenflussanalyse Schritte zurückzugehen und sich einen vorherigen Analysezustand erneut anzeigen zu lassen.

\criteriumOptional{Auswahlmöglichkeit der Worklistalgorithmen - Priorität 2}{crt:WorklistO}
Der Benutzer kann zwischen verschiedenen Worklistalgorithmen wählen. Diese
entscheiden, in welcher Reihenfolge die Blöcke in der Datenflussanalyse behandelt
werden.

\criteriumOptional{.java-File mit mehreren Funktionen - Priorität 3}{crt:moreFncO}
Der Benutzer kann Java-Code einlesen oder eingeben, welcher mehrere Funktionen
enthält.

\criteriumOptional{\glqq Jump to Action\grqq -Funktion - Priorität 2}{crt:JumpToActionO}
Der Benutzer hat die Möglichkeit den aktuell analysierten Block beziehungsweise die aktuell analysierte Zeile als seinen ausgewählten Block beziehungsweise seine ausgewählte Zeile zu setzen.

\criteriumOptional{Hotkeys für Datenflussanalyse - Priorität 3}{crt:DFAo}
Dem Benutzer steht eine Reihe von Hotkeys zur Navigation durch die
Datenflussanalyse zur Verfügung.

\criteriumOptional{Weitere Datenflussanalysen - Priorität 2}{crt:DFAArtenO}
Der Benutzer kann zur Analyse seines eingegebenen oder eingelesenen Java-Codes
zusätzlich zu den Datenflussanalysen aus M7 aus weiteren Arten der Datenflussanalyse
wählen.

\criteriumOptional{Färbung des CFG - Priorität 1}{crt:HighlightingO}
Die Blöcke des angezeigten CFG im Graph-Panel werden in
Abhängigkeit des aktuellen Analysestandes in verschiedenen Farben eingefärbt.

\criteriumOptional{Exportieren des CFG als Bild-Datei - Priorität 3}{crt:exportPicO}
Der Benutzer hat die Möglichkeit, den aktuell im Graph-Panel angezeigten CFG als Bilddatei zu exportieren. 

\criteriumOptional{Manual Page - Priorität 1}{crt:ManPageO}
Dem Programm liegt eine Manual Page bei, in welcher der Benutzer über die wichtigsten Funktionen des Programms informiert wird.

\criteriumOptional{Fortschrittsanzeige - Priorität 2}{crt:ProgressBarO}
Dem Benutzer wird angezeigt, wie weit die Datenflussanalyse fortgeschritten ist.

\criteriumOptional{Springen zu beliebigem Analysepunkt - Priorität 2}{crt:SliderO}
Der Benutzer kann an einen beliebigen Punkt der Datenflussanalyse springen.

\criteriumOptional{Implementieren einer eigenen Datenflussanalyse - Priorität 3}{crt:OwnAnalyseO}
Der Benutzer kann den CFG mit einer eigenen Datenflussanalyse analysieren.

\newpage
\subsection{Abgrenzung}

\criteriumNot{Kein Tutorial}{crt:noTutorial}
Das Programm erklärt nicht, wie Datenflussanalysen funktionieren oder was ein CFG ist.

\criteriumNot{Keine Optimierung}{crt:noOptimize}
Das Programm ist zur Veranschaulichung von Datenflussanalysen gedacht. Es kann keine Programme optimieren.

\criteriumNot{Funktionalität}{crt:functionality}
Werden folgende Grenzen überschritten, so ist die Funktionalität des Programms nicht mehr garantiert:
\begin{enumerate}[label=(\alph*)]
\item Der CFG hat maximal 10 Blöcke
\item Jeder Block beinhaltet maximal 10 Zeilen Zwischencode
\item Die Datenflussanalyse benötigt maximal 5000 (Zeilen-)Schritte
\item Der eingegebene Java-Code besteht aus maximal 5000 Zeichen
\end{enumerate}