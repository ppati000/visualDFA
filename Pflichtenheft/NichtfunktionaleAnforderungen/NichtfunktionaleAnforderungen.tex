%note: don't split this document up with include{...}

\section{Nichtfunktionale Anforderungen}

\nonFunctionality{Benutzeroberfläche reagiert zu jedem Zeitpunkt}{nfc:reactivity}

Die Benutzeroberfläche reagiert in jedem Zustand des Programms (also auch während der Berechnung der Datenflussanalyse) innerhalb von einer Sekunde auf Eingaben des Nutzers.

\nonFunctionality{Visuelles Feedback}{nfc:feedback}

Nach Betätigung eines Buttons erhält der Nutzer innerhalb von einer Sekunde visuelles Feedback.

\nonFunctionality{Konsistenz zwischen Code und Graph}{nfc:consistency}

Der eingelesene oder eingegebene Java-Code ist zu jedem Zeitpunkt konsistent zum angezeigten Kontrollflussgraphen, sofern ein Kontrollflussgraph angezeigt wird.

\nonFunctionality{Aussagekräftige Fehlermeldungen}{nfc:errors}

Java-Kenntnisse und ein Verständnis für das Konzept von Datenflussanalysen vorausgesetzt, enthalten die durch das Programm angezeigten Fehlermeldungen sowohl zum eingegebenen Java-Code als auch zu den Einstellungen des Programms in Kombination mit der Manual Page ausreichend Informationen, dass die Ursache des jeweiligen Fehlers präzise festgestellt werden kann.

\nonFunctionality{Garantierte Funktionalität}{nfc:guarantees}

Die Funktionalität des Programmes ist garantiert, solange keine der folgenden Grenzen überschritten wird:
\begin{enumerate}[label=(\alph*)]
\item Der Kontrollflussgraph hat maximal 10 Knoten
\item Jeder Knoten beinhaltet maximal 10 Zwischencode-Zeilen
\item Die Datenflussanalyse benötigt maximal 5000 (Zeilen-)Schritte
\item Der eingegebene Java-Code besteht aus maximal 5000 Zeichen
\end{enumerate}

\nonFunctionality{Skalierbarkeit der GUI}{nfc:scalability}

Die GUI ist beliebig skalierbar für Fenstergrößen größer oder gleich der geforderten Mindestauflösung (siehe Produktumgebung).

\nonFunctionality{Äquivalenz von Einlesen und Eingabe}{nfc:equivalence}

Das Programm behandelt direkt eingegebenen Java-Code vollständig äquivalent zu aus einer .java Datei eingelesenem Java-Code.
