\documentclass[parskip=full,11pt]{scrartcl}

\usepackage[utf8]{inputenc}
\usepackage[T1]{fontenc}
\usepackage[german]{babel}

\usepackage[yyyymmdd]{datetime} % must be after babel
\renewcommand{\dateseparator}{-} % ISO8601 date format

\usepackage{hyperref}
\usepackage{csquotes}
\usepackage{glossaries}

\usepackage[nameinlink]{cleveref}

\usepackage{xcolor}
\usepackage{graphicx}
\hypersetup{
	pdftitle={Pflichtenheft}
}

\usepackage{pflichtenheft}

\usepackage{float}
\usepackage{enumitem}

% we might want to reconsider this but by dafault there is a LOT of white space at the end of pages
\usepackage[top=2cm, bottom=3cm, left=2cm, right=2cm]{geometry}


\usepackage{listings}
\lstset{
	mathescape,
	numbers=left, 
	numberstyle=\tiny, 
	breaklines=true,
	numbersep=5pt,
	xleftmargin=10pt,
	xrightmargin=5pt,
	tabsize=2} 

\usepackage{tikz}
\usetikzlibrary{positioning,shapes.geometric}

\makeindex
\makeglossaries

% so wird ein Glossareintrag definiert
%\newglossaryentry{computer}{
%	name=computer,
%	description={is a programmable machine that receives input,
%		stores and manipulates data, and provides
%		output in a useful format}
%}

\begin{document}
	\title{Pflichtenheft: Programmanalyse zum Durchklicken}
	\author{Nils Jessen \and Anika Nietzer \and Patrick Petrovic \and Sebastian Rauch \and Michael Schieber}
	
	\maketitle


	\part{Einleitung}

Dieses Dokument beschreibt die Implementierung des Programms \enquote{Datenflussanalyse zum Durchklicken}. Zunächst wird der Implementierungsplan vorgestellt, welcher die vorgesehene Aufgabenteilung enthält.
Darauf folgt ein Bericht über die Ausführung des Plans. Dabei wird insbesondere darauf eingegangen, welcher Funktionsumfang implementiert wurde.
In einem gesonderten Kapitel wird beschrieben, welche Werkzeuge während der Implementierung verwendet wurden, um verlässliche Organisation und Kommunikation innerhalb des Teams sicherzustellen.
Abschließend folgt eine Zusammenstellung der Entwurfsänderungen, welche sich im Laufe der Implementierung ergeben haben. 
	%note: don't split this document up with include{...}

\section{Kriterien}

\subsection{Muss}

\criterium{Eingabe von Code}{crt:CodeInput}
Eingegebener Java-Code kann zu einem Kontrollflussgraphen (CFG) verarbeitet werden.

\criterium{Java-Subset}{crt:minSubset}
Das Programm benutzt eine minimale Teilmenge an Java-Operatoren, Kontrollflussstrukturen und Datentypen der Java Version 1.7.

\criterium{Kontrollflussgraph}{crt:CFG}
Generierter Zwischencode wird dem Benutzer im Kontrollflussgraphen (CFG) angezeigt.

\criterium{Datenflussanalysen}{crt:DFA}
Das Programm beherrscht mehrere Datenflussanalysen, mit welchen es den Kontrollflussgraphen analysieren kann.
Diese sind:\par
\begin{enumerate}[label=(\alph*)]
\item Constant Folding
\item Reaching Definitions
\item Constant Bits
\item Taint Analysis.
\end{enumerate}

\criterium{Anzeigen der Analyse}{crt:ShowAnalysis}
Die Datenflussanalysen können dem Benutzer schrittweise animiert im Kontrollflussgraph angezeigt werden.

\criterium{Fixpunkt}{crt:Fixpunkt}
Das Programm kann den Fixpunkt der Analyse automatisch berechnen.

\criterium{Abbruch der Analyse}{crt:AbortAnalysis}
Der Benutzer kann die Analyse abbrechen und zur Code-Eingabe zurückkehren.

\subsection{Kann}

\criteriumOptional{Erweitertes Java-Subset 1}{crt:extendedSubset1o}
Das Programm kann Java-Code einlesen und verarbeiten, wenn dieser folgende Operatoren und Konstrukte beinhaltet:

Modulo Operator: \%

Bit-shift Operatoren: <<, >>

Logisches \glqq Und\grqq und \glqq Oder\grqq : \&, |

\criteriumOptional{Erweitertes Java-Subset 2}{crt:extendedSubset2o}
Das Programm kann Java-Code einlesen und verarbeiten, wenn dieser folgende Operatoren und Konstrukte beinhaltet:

for - Schleife (nicht for each)

else if - Bedingung

\criteriumOptional{Erweitertes Java-Subset 3}{crt:extendedSubset3o}
Das Programm kann Java-Code einlesen und verarbeiten, wenn dieser folgende Operatoren und Konstrukte beinhaltet:

Zuweisungen: +=, -=, *=, /=, \% =

\criteriumOptional{Zeilenangabe bei Fehlermeldungen}{crt:lineNmbrErrorO}
Entstehen beim Kompilieren des eingelesenen Codes Fehlermeldungen, so wird für den Nutzer ersichtlich, in welcher Zeile seines Java-Codes er den Fehler gemacht hat.

\criteriumOptional{Breakpoints in Datenflussanalyse}{crt:BreakpointsO}
Der Nutzer kann in seiner Datenflussanalyse Breakpoints setzen. Beim automatisierten Durchlaufen der Analyse stoppt diese am nächsten gesetzten Breakpoint.

\criteriumOptional{Analyse einzelner Zeilen innerhalb der Knoten}{crt:lineAnalysisO}
Beim Durchlaufen der Datenflussanalyse erhält der Nutzer nicht nur In- und Out-
Informationen zu den Knoten als Ganzes sondern auch zu den einzelnen Zeilen
innerhalb der Knoten.

\criteriumOptional{Dynamisches Graph-Layout}{crt:GraphLayoutO}
Der Nutzer hat die Möglichkeit das zunächst automatisierte Layout des
Kontrollflussgraphen selbst zu verändern, in dem er in den Graphen rein- und
rauszoomt und in dem er die einzelnen Knoten hin und her beweget.

\criteriumOptional{Einlesen von Code aus .java Datei}{crt:ComdeInputExto}
Der Nutzer hat die Möglichkeit, eine .java Datei in das Programm zu laden, welche
den zu analysierenden Code enthält. Den eingelesenen Code kann er anschließend
auch im Programm bearbeiten.

\criteriumOptional{Delay-Optionen zwischen Autoplay-Schritten}{crt:DelayO}
Der Nutzer kann einstellen, wie lange die einzelnen Schritte im automatisierten
Durchlaufen der Datenflussanalyse angezeigt werden sollen. Hierbei reicht der
Optionsumfang von 0 Sekunden Delay (so schnell wie möglich zum Fixpunkt) bis hin
zu einigen Sekunden Anzeigezeit pro Schritt.

\criteriumOptional{Rückwärtsschritte in der Datenflussanalyse}{crt:backwardsAnalysisO}
Der Nutzer hat die Möglichkeit, in der Datenflussanalyse Schritte zurück zu gehen
und sich einen vorherigen Analysezustand erneut anzeigen zu lassen.

\criteriumOptional{Auswahlmöglichkeit der Worklistalgorithmen}{crt:WorklistO}
Der Nutzer kann zwischen verschiedenen Worklistalgorithmen wählen. Diese
entscheiden, in welcher Reihenfolge die Knoten in der Datenflussanalyse behandelt
werden. Es stehen folgende Algorithmen zur Auswahl:
\begin{enumerate}[label=(\alph*)]
\item \textbf{Naiv} \par
Von oben nach unten, von links nach rechts werden all Knoten in der Reihenfolge in
die Worklist geschrieben, in welcher sie \glqq entdeckt\grqq werden und dann in dieser Reihenfolge abgearbeitet.
\item \textbf{Random} \par
Die Blöcke werden wie beim naiven Worklistalgorithmus auf die Worklist geschrieben. Am Ende der Abarbeitung jedes Knotens in der Analyse wird ein
zufälliger neuer dieser Blöcke gewählt und behandelt.
\item \textbf{Efficient} \par
Die Reihenfolge der Abarbeitung der Blöcke wird optimiert, indem beispielsweise
zuerst innere Schleifen durchlaufen werden, bis ein lokaler Fixpunkt gefunden wurde, bevor die äußere Schleife weiter analysiert wird.
\end{enumerate}

\criteriumOptional{Speichern in .java Datei}{crt:saveO}
Der Nutzer kann den eingelesenen oder eingetippten Java-Code als .java Datei
speichern.

\criteriumOptional{.java File mit mehreren Funktionen}{crt:moreFncO}
Der Nutzer kann Java-Code einlesen oder eingeben, welcher mehrere Funktionen
enthält. Vor der Datenflussanalyse kann er dann auswählen, welche Methode
analysiert werden soll.

\criteriumOptional{\glqq Jump to action\grqq -Funktion}
Der Nutzer hat die Möglichkeit mit nur einem Mausklick den aktuell analysierten
Knoten beziehungsweise die aktuell analysierte Zeile als seinen ausgewählten Knoten beziehungsweise seine ausgewählte Zeile zu setzen, egal welchen anderen Knoten und in welcher Zeile im Graphen er sich zuvor befunden hat. Dabei wird auch die Ansicht automatisch auf den fraglichen Knoten ausgerichtet.

\criteriumOptional{Hotkeys für Datenflussanalyse}{crt:DFAo}
Dem Nutzer steht eine Reihe von Hotkeys zur Navigation durch die
Datenflussanalyse zur Verfügung.

\criteriumOptional{Weitere Datenflussanalysearten}{crt:DFAArtenO}
Der Nutzer kann zur Analyse seines eingegebenen oder eingelesenen Java-Codes
zusätzlich zu den Analysearten aus M7 aus 3 weiteren Arten der Datenflussanalyse
auswählen:
\begin{enumerate}[label=(\alph*)]
\item  Live Variable Analysis
\item Definite Assignment Analysis
\item Available Expression Analysis
\end{enumerate}

\criteriumOptional{Färbung des Graphen}{crt:HighlitingO}
Die Knoten des angezeigten Kontrollflussgraphen im Graph-Panel werden in
Abhängigkeit des aktuellen Analysestandes in 4 verschiedenen Farben eingefärbt:
\begin{enumerate}[label=(\alph*)]
\item Unvisited
\item On Worklist
\item In Action
\item Visited and not on Worklist
\end{enumerate}

\criteriumOptional{Exportieren des Graphen als Bild-Datei}{crt:exportPicO}
Der Nutzer hat die Möglichkeit den aktuell im Graph-Panel angezeigten Kontrollflussgraphen als Bilddatei zu exportieren. Er kann auch automatisiert mehrere Analyseschritte der Datenflussanalyse als Batch-Export exportieren.

\criteriumOptional{\glqq Speichern?\grqq -Meldung vor schließen}{crt:unsavedChangesO}
Der Nutzer wird gefragt, ob ungespeicherte Änderungen des .java files im Textfeld
speichern möchte, wenn er das Programm schließt und ungespeicherte Änderungen vorliegen.

\criteriumOptional{\glqq Ungespeicherte Änderungen verfallen\grqq -Meldung bei Laden einer neuen Datei}{crt:loadUnsavedO}
Der Nutzer bekommt eine Meldung, dass ungespeicherte Änderungen verfallen,
wenn eine neue .java Datei im Textfeld geladen werden soll und ungespeicherte
Änderungen vorliegen.

\criteriumOptional{Manual Page}{crt:ManPageO}
Dem Programm liegt eine Manual Page bei, in welcher der Benutzer über die wichtigsten Funktionen des Programms informiert wird.
	%note: don't split this document up with include{...}

\section{Produkteinsatz}

Das Produkt ist für die Lehre gedacht. Es wird erwartet, dass Benutzer über grundlegende Kenntnisse von Kontrollflussgraphen und Datenflussanalysen verfügen.
Außerdem sollten sich Benutzer grundlegend mit der Programmiersprache Java auskennen.

Es ist dafür gedacht, von Lehrpersonal und Studenten benutzt zu werden.
	%note: don't split this document up with include{...}

\section{Produktumgebung}

... hier die Produktumgebung ...
	%note: don't split this document up with include{...}

\section{Funktionale Anforderungen}

\functionality{Zeilennummerierung}{fnc:LineNumbers}
Im Textfeld für die Eingabe des Codes gibt es eine Zeilennummerierung.

\functionality{Code Eingabe mit Editor}{fnc:CodeInputIntern}
\fulfills{crt:CodeInput}
Es gibt ein Textfeld in welches der Benutzer Code eingeben kann, welcher dann zu einem Kontrollflussgraphen (CFG) verarbeitet werden kann.

\functionality{Datei Eingabe}{fnc:CodeInputExt}
\fulfills{crt:CodeInputExto}
Benutzer kann einzulesenden Code aus einer .java Datei laden.

\functionality{Java-Subset}{fnc:minimalSubset}
\fulfills{crt:minSubset}
Das Programm unterstützt ein minimales Subset von Java.
Unterstützt werden:
\begin{enumerate}[label=(\alph*)]
\item Die Arithmetischen Operationen +, -, *, /,
\item die Kontrollflussstrukturen if, else, while,
\item die Vergleichsoperatoren <, >, ==,
\item die Datentypen Integer und Byte,
\item die Zuweisung =.
\end{enumerate}

\functionality{Erweitertes Subset 1}{fnc:extendedSubset1}
\fulfills{crt:extendedSubset1o}
Das Programm unterstützt zusätzlich zum minimalen Subset noch:
Die Operatoren \%, <<, >>, |, \&.

\functionality{Erweitertes Subset 2}{fnc:extendedSubset2}
\fulfills{crt:extendedSubset2o}
Das Programm unterstützt zusätzlich zum minimalen Subset noch:
For-Schleifen und else-if.

\functionality{Erweitertes Subset 3}{fnc:extendedSubset3}
\fulfills{crt:extendedSubset3o}
Das Programm unterstützt zusätzlich zum minimalen Subset noch:
Die Zuweisungen +=, -=, *=, /=, \%=.

\functionality{Syntaxfehler}{fnc:syntaxError}
\fulfills{crt:minSubset}
\fulfills{crt:lineNmbrErrorO}
Enthält der vom Benutzer eingegebene Code Syntaxfehler, so erhält der Benutzer eine Fehlermeldung, welche bezüglich des Informationsgehaltes der Fehlermeldung des Compilers entspricht.

\functionality{Nichtunterstützte Operationen}{fnc:OpsError}
\fulfills{crt:minSubset}
\fulfills{crt:lineNmbrErrorO}
Der Benutzer bekommt eine Fehlermeldung, wenn in der zu analysierenden Funktion Operationen und Kontrollflusskonstrukten vorhanden sind welche nicht dem definierten Subset von Java entsprechen. Die Fehlermeldung weißt auf das unterstütze Java-Subset hin.

\functionality{Anzeigen des Kontrollflussgraphen (CFG)}{fnc:showCFGf}
\fulfills{crt:CFG}
Die GUI hat kann den aus dem Java-Code erstellten Graphen in einem Zwischencode anzeigen. Dieser Graph ist in einem Panel dargestellt.

\functionality{Interaktion mit dem Kontrollflussgraphen 1}{fnc:interactCFG1f}
\fulfills{crt:GraphLayoutO}
\fulfills{crt:BreakpointsO}
Der Benutzer kann mit der GUI des CFG interagieren, um Breakpoints zu setzen.

\functionality{Interaktion mit dem Kontrollflussgraphen 2}{fnc:interactCFG2f}
\fulfills{crt:GraphLayoutO}
Der Benutzer kann mit der GUI des CFG interagieren, um zu scrollen und zu zoomen.

\functionality{Interaktion mit dem Kontrollflussgraphen 3}{fnc:interactCFG3f}
\fulfills{crt:GraphLayoutO}
Der Benutzer kann mit der GUI des CFG interagieren, um Blöcke, beziehungsweise Zeilen auszuwählen.

\functionality{Auswahl der Analysearten}{fnc:pickAnalysis}
\fulfills{crt:DFA}
\fulfills{crt:DFAArtenO}
Der Benutzer kann zwischen allen implementierten Analysearten auswählen und sie auf den Graph anwenden.

\functionality{Blockweise durchlaufen der Analyse}{fnc:blockAnalysis}
\fulfills{crt:ShowAnalysis}
\fulfills{crt:backwardsAnalysisO}
Der Benutzer kann die Analyse blockweise durchlaufen. Dafür stehen ihm Buttons zur Verfügung.\par
\faFastForward\ : Springt einen Block weiter.\par
\faFastBackward\ : Springt einen Block zurück.

\functionality{Zeilenweise durchlaufen der Analyse}{fnc:lineAnalysis}
\fulfills{crt:lineAnalysisO}
\fulfills{crt:backwardsAnalysisO}
Der Benutzer kann die Analyse zeilenweise durchlaufen. Dafür stehen ihm Buttons zur Verfügung.\par
\faForward\ : springt eine Zeile weiter.\par
\faBackward\ : springt eine Zeile zurück.

\functionality{Abbrechen der Analyse}{fnc:stopAnalysisf}
\fulfills{crt:AbortAnalysis}
Der Benutzer kann eine laufende Analyse abbrechen, um neuen Code analysieren zu lassen.

\functionality{Automatisches durchlaufen der Analyse}{fnc:automaticAnalysis}
\fulfills{crt:Fixpunkt}
Der Benutzer kann den \faPlay\ Button betätigen, um die Analyse automatisch bis zum Fixpunkt Durchzulaufen.

\functionality{Automatisches durchlaufen bis Breakpoint}{fnc:breakAnalysisf}
\fulfills{crt:BreakpointsO}
Die Analyse kann den \faPlay\ Button betätigen, um automatisch bis zu einem gesetzten Breakpoint zu laufen.

\functionality{Worklistalgorithmen}{fnc:worklistNaive}
\fulfills{crt:Fixpunkt}
Das Programm kann zur Fixpunktanalyse den naiven Worklistalgorithmus anwenden.

\functionality{Erweiterte Worklistalgorithmen}{fnc:moreWorklist}
\fulfills{crt:WorklistO}
Der Nutzer kann zwischen verschiedenen Worklistalgorithmen wählen. Diese
entscheiden, in welcher Reihenfolge die Knoten in der Datenflussanalyse behandelt
werden. Es stehen folgende Algorithmen zur Auswahl:
\begin{enumerate}[label=(\alph*)]
\item \textbf{Naiv} \par
Von oben nach unten, von links nach rechts werden all Knoten in der Reihenfolge in
die Worklist geschrieben, in welcher sie „entdeckt“ werden und dann in dieser
Reihenfolge abgearbeitet.
\item \textbf{Random} \par
Die Blöcke werden wie naiven Worklistalgorithmus auf die Worklist geschrieben. Am Ende der Abarbeitung jedes Knotens in der Analyse wird ein
zufälliger neuer dieser Blöcke gewählt und behandelt.
\item \textbf{Efficient} \par
Die Reihenfolge der Abarbeitung der Blöcke wird optimiert, indem beispielsweise
zuerst innere Schleifen durchlaufen werden, bis ein lokaler Fixpunkt gefunden wurde, bevor die äußere Schleife weiter analysiert wird.
\end{enumerate}

\functionality{Code speichern}{fnc:saveCode}
\fulfills{crt:saveO}
Der Benutzer kann den Inhalt des Textfeldes zur Codeeingabe in einer .java Datei abspeichern.

\functionality{Hotkeys}{fnc:Hotkeys}
\fulfills{crt:DFAo}
Dem Benutzer stehen Hotkeys zur Verfügung mit Hilfe derer er die Datenflussanalyse steuern kann.

\functionality{Optisches Hervorheben der Blöcke}{fnc:Highlights}
\fulfills{crt:HighlightingO}
Je nach Status werden die Blöcke im Kontrollflussgraphen farbig hervorgehoben.
\begin{enumerate}[label=(\alph*)]
\item Es gibt die Zustände:
\item Noch nicht besucht.
\item Auf der Worklist.
\item Gerade ausgewählt.
\item Bereits besucht und nicht mehr auf der Worklist.
\end{enumerate}

\functionality{Eingangs- und Ausgangszustand von Blöcken}{fnc:inOutBlockf}
Dem Benutzer werden Eingangs- und Ausgangszustand des ausgewählten Blocks in einem GUI-Element angezeigt.

\functionality{Eingans- und Ausgangszustand von Zeilen}{fnc:inOutLinef}
Dem Benutzer werden Eingangs- und Ausgangszustand der ausgewählten Zeile in einem GUI-Element angezeigt.

\functionality{Graphen exportieren}{fnc:exportAsPicf}
\fulfills{crt:exportPicO}
Der Benutzer kann von allen Schritten der Analyse ein Bild des Graphen abspeichern.

\functionality{Delay-Optionen}{fnc:Delayf}
\fulfills{crt:DelayO}
Der Nutzer kann zwischen den automatischen Schritten eine Zeitverzögerung einstellen. Diese reicht von 0 Sekunden künstlichem Delay (so schnell es geht) bis hin zu einigen Sekunden Verzögerung.

\functionality{Ungespeicherte Änderungen}{fnc:unsavedChangesf}
\fulfills{crt:unsavedChangesO}
\fulfills{crt:loadUnsavedO}
Gibt es ungespeicherte Änderungen im Editor, wenn der Benutzer das Programm schließen möchte, so wird er gefragt, ob er diese speichern möchte.
Falls der Benutzer eine neue .java Datei öffnen möchte und ungespeicherte Änderungen vorliegen, so wird der Benutzer gewarnt, dass diese Änderungen verfallen.

\functionality{Mehrere Funktionen}{fnc:moreFncf}
\fulfills{crt:moreFncO}
Der Nutzer kann Java-Code einlesen oder eingeben, welcher mehrere Funktionen
enthält. Liegen mehrere Funktionen vor, so fragt das Programm vor Start der Analyse welche Funktion analysiert werden soll.

\functionality{Jump to Action}{fnc:jumpToActionf}
\fulfills{crt:JumpToActionO}
Der Benutzer kann mit Hilfe eines Buttons das automatische Springen zum aktiven Block, beziehungsweise zur aktiven Zeile aktivieren. Dieses bleibt so lange aktiv bis er den Button erneut betätigt wird.

\functionality{Manual Page}{fnc:ManPagef}
\fulfills{crt:ManPageO}
Dem Programm liegt eine Manual Page bei, in welcher der Benutzer über die wichtigsten Funktionen des Programms informiert wird.

\functionality{Fortschrittsanzeige}{fnc:ProgressBarf}
\fulfills{crt:ProgressBarO}
Das Programm besitzt einen Fortschrittsbalken. Dieser zeigt zu jedem Zeitpunkt der Analyse, wie viele Schritte im Verhältnis zur Gesamtzahl der Schritte bereits erfolgt sind.

\functionality{Springen zu beliebigem Analysepunkt}{fnc:Sliderf}
\fulfills{crt:SliderO}
Das Programm besitzt einen Slider mit welchem der Benutzer zu jedem Punkt der Analyse springen kann.
	%note: don't split this document up with include{...}

\section{Nichtfunktionale Anforderungen}

... hier die nichtfunktionalen Anforderungen ...
	%TODO Anführungszeichen richtig machen
\section{Tests (Allgemein)}

\test{Interaktion mit dem Programm unabhängig von aktueller Analyse}{tst:general}
\tests{}

\teststep{Der Benutzer hat das Programm noch nicht geöffnet.}
{Der Benutzer klickt auf die entsprechende Ausführungsdatei.}
{Das Programm öffnet sich mit den definierten GUI Eigenschaften.}

\teststep{Der Benutzer befindet sich an einer beliebigen Stelle des Arbeitsflusses.}
{Der Nutzer ändert die Fenstergröße.}
{Der Inhalt des Fensters wird dynamisch angepasst.}

\teststep{[opt] Der Benutzer befindet sich an einer beliebigen Stelle des Arbeitsflusses.}
{Der Benutzer ruft über den Hilfebutton die Hilfefunktion auf.}
{Die Manual Page öffnet sich, welche die wichtigsten Funktionen unseres Programms kurz zusammenfasst.}

\teststep{[opt]Der Benutzer befindet sich an einer beliebigen Stelle des Arbeitsflusses.}
{Er klickt entweder rechts oder links auf einen Pfeil/Button, der die Leiste verkleinern soll.}
{Die entsprechend ausgewählte Leiste wir ausgeblendet und das Graph-Panel wird um den frei gewordenen Platz erweitert.}

\teststep{[opt]Der Benutzer befindet sich an einer beliebigen Stelle des Arbeitsflusses. Im Editor steht Java-Code der bisher nicht gespeichert wurde.}
{Der Benutzer klickt auf einen Button, der zum schließen des Programms führt.}
{In einem Fenster wird darauf hingewiesen, dass die Änderungen, bei Fortsetzung der Aktion, verloren gehen und eine Möglichkeit, den Vorgang abzubrechen wird gegeben.}

\test{Importieren, schreiben und speichern von Code für die Analyse}{test:code}
\tests{}

\teststep{Der Benutzer befindet sich nicht in einer aktuellen Analyse und hat noch keinen Java-Code für die Analyse zur Verfügung.}
{Der Benutzer tippt den zu analysierenden Code in ein Textfeld.}
{Dieser Code steht für die Analyse zur Verfügung.}

\teststep{[opt] Der Benutzer befindet sich nicht in einer aktuellen Analyse und hat noch keinen Java-Code für die Analyse zur Verfügung.}
{\glqq Open...\grqq\ wird geklickt.}
{Ein File-Chooser öffnet sich und der Benutzer kann in seinem Dateisystem nach der zu verwendenden Java-Datei suchen.}

\teststep{[opt] Der Benutzer befindet sich nicht in einer aktuellen Analyse und hat nicht gespeicherten Code in seinem Editor.}
{\glqq Open...\grqq\ wird geklickt.}
{In einem Fenster wird darauf hingewiesen, dass die Änderungen bei Fortsetzung der Aktion verloren gehen und eine Möglichkeit, den Vorgang abzubrechen wird gegeben. Wenn kein Abbruch erfolgt, öffnet sich der File-Chooser.}

\teststep{[opt]Der File-Chooser ist geöffnet.}
{Der Benutzer öffnet eine Directory die keine Datei mit der Endung .java enthält.}
{Es werden keine Dateien angezeigt.}

\teststep{[opt] Der File-Chooser ist geöffnet.}
{Eine korrekte Java-Datei wird zum importieren ausgewählt.}
{Der File-Chooser schließt sich und die Datei wird im Textfeld angezeigt.}

\teststep{[opt]Ein Code-Stück ist im Textfeld dargestellt.}
{\glqq Save... \grqq\ wird geklickt.}
{Ein Fenster öffnet sich, in dem der Nutzer den Namen der Datei und ihren Speicherort auswählen kann. Durch klicken des \glqq Speicher\grqq -Buttons wird die Datei gespeichert und das Fenster schließt sich wieder.}


\test{Auswahl der Analyseart}{tst:anaopt}
\tests{}

\teststep{Der Benutzer befindet sich nicht in einer aktuellen Analyse.}
{Der Benutzer wählt eine Programmanalyseart aus.}
{Diese Analyseart wird bei dem nächsten Aufruf von \glqq Start Analysis \grqq\ verwendet.}

\teststep{[opt] Der Benutzer befindet sich nicht in einer aktuellen Analyse.}
{Der Nutzer wählt einen Worklist-Algorithmus aus.}
{Diese Art von Worklist-Algorithmus wird beim nächsten Aufruf von \glqq Start Analysis \grqq\ verwendet.}


\test{Erstellen eines Kontrollflussgraphen}{tst:creategraph}
\tests{}

\teststep{[ggf. opt] Ein Code-Stück, welches nur gültige Java-Syntax aus dem von uns unterstützen Java-Subset enthält, ist im Textfeld dargestellt. Es besteht aus nur einer Methode.}
{\glqq Start Analysis \grqq\ wird geklickt}
{Der Kontrollflussgraph wird im Graph-Panel dargestellt und die Animation kann jetzt mit \faStepForward\ oder \faPlay\ gestartet werden. Eine Änderung der Analyseart oder des Codes ist nur noch durch einen Abbruch mit \faStop\ möglich.}

\teststep{Ein Code-Stück ist im Textfeld dargestellt, das entweder nicht gültige Java-Syntax enthält, oder solchen Java-Code, dessen übersetzter Zwischencode nicht identisch zum Zwischencode eines Parts unseres unterstützen Java-Subsets ist.}
{\glqq Start Analysis \grqq\ wird geklickt.}
{Der Graph wird nicht erstellt, eine aussagekräftige Fehlermeldung wird ausgegeben, in der der Java-Error oder ein Hinweis auf das von uns unterstützte Java-Subset gegeben wird.}

\teststep{[opt]Ein Code-Stück ist im Textfeld dargestellt, das gültige Java-Syntax aus dem von uns unterstützen Java-Subset enthält. Es besteht aus mehreren Methoden.}
{\glqq Start Analysis \grqq\ wird geklickt.}
{Ein Fenster öffnet sich, in dem die Methode ausgewählt werden soll, für welche der Kontrollflussgraph erstellt und die Analyse durchgeführt werden soll.}


\test{Interaktion mit dem Kontrollflussgraphen}{tst:intcfg}
\tests{}

\teststep{[opt]Ein Kontrollflussgraph wurde erstellt.}
{Der Benutzer scrollt mit dem Mausrad nach oben bzw. unten, während sich der Mauszeiger im Graph-Panel befindet.}
{Der gesamte Graph wird entsprechend skaliert. Falls der Graph nicht vollständig im Graph-Panel angezeigt werden kann erscheinen Scroll-Leisten.}

%Wirklich?, habe das jetzt mal so festgelegt, müsste eventuell noch in die funktionalen Anforderungen mit aufgenommen werden
\teststep{[opt]Ein Graph wurde erstellt.}
{Der Benutzer verschiebt per \glqq Drag and Drop \grqq einen Knoten.}
{Der Knoten verschiebt sich an die entsprechende Stelle und alle Kanten werden entsprechend mit verschoben. Überlappen sich nun 2 Knoten befindet sich der Knoten, welcher zuletzt verschoben wurde, im Vordergrund.}

\test{Bildexport}{tst:graph}
\tests{}

\teststep{[opt]Ein Graph wurde erstellt.}
{\glqq Graph exportieren \grqq\. wird geklickt.}
{Ein Fenster öffnet sich, in dem der Benutzer wählen kann, ob er den aktuell angezeigten Zwischenschritt oder alle Zwischenschritte speichern möchte und gegebenenfalls, ob diese zeilenweise oder blockweise gespeichert werden. Außerdem kann er hier zwischen der Auflösung \glqq low \grqq\, \glqq standard \grqq\ und  \glqq  high \grqq\ wählen. Nach bestätigen dieser Eingabe öffnet sich ein File-Explorer, in dem der Name und der Speicherort der Datei festgelegt werden kann. Alle Bilder werden als .png gespeichert. Nach Bestätigung dieser Eingabe mit  \glqq Speichern \grqq\ wurde das Bild dort abgelegt.}


\test{Animationsmöglichkeiten}{tst:aniposs}
\tests{}


\teststep{Der \glqq Start Analysis \grqq\ Button wurde geklickt.}
{Es wird \faStop\ geklickt.}
{Der Nutzer wird gefragt ob er seine Analyse wirklich abbrechen möchte und wird darüber informiert, dass dieser Schritt den bisherigen Graph löscht. Bei Bestätigung dieser Meldung wird der Graph gelöscht und das Graph-Panel und Animations-Button ausgegraut. Eine Änderung des Codes und der Analyseart sind jetzt wieder möglich.}


\test{Animation in Zusammenhang mit der Delay-Auswahl}{tst:delay}
\tests{}

\teststep{Es wurde in der Zeitauswahl 0 Sekunden als Delay ausgewählt und es wurden keine Breakpoints gesetzt.}
{Es wird \faPlay\ geklickt.}
{Der Fixpunkt wird so schnell wie möglich angezeigt, alle Zwischenschritte werden in der Darstellung übersprungen.}

\teststep{[opt]Es wurde in der Zeitauswahl ein von 0 verschiedenes Delay ausgewählt, es wurden keine Breakpoints gesetzt.}
{Es wird der Button \faPlay\ geklickt.}
{Die einzelnen Schritte der Analyse werden mit einem zeitlichen Abstand, der dem ausgewählten Delay entspricht, dargestellt. Die Farben der Blöcke ändern sich je nach Zustand zwischen den verschiedenen Schritten. Zusätzlich ändert sich bei jedem Schritt die linke Leiste, welche \glqq in \grqq\ und \glqq out \grqq\ des aktuellen Schritts, sowie den aktuellen Schritt selbst anzeigt.}

\teststep{[opt]Es wurden Breakpoints innerhalb des Graphen gesetzt.}
{Es wird der Button \faPlay\ geklickt.}
{Die Analyse läuft schrittweise mit dem jeweiligen Delay ab, wobei garantiert wird, dass bei jedem Erreichen des Breakpoints die Analyse automatisch pausiert wird.}

\teststep{[opt]Die Analyse wurde gestartet und der \glqq jump to action \grqq\ Haken ist gesetzt.}
{Es wird der Button \faPlay\ geklickt.}
{Bei jedem neuen Analyseschritt, der angezeigt wird, wird der Graph so zentriert, dass der aktuell analysierte Knoten in der Mitte liegt. Passt der Graph komplett in das Graph-Panel wird nichts verändert.}


\test{Animation bei schrittweiser Berechnung}{tst:calcpart}
\tests{}

\teststep{Der \faPlay\ Button ist gedrückt.}
{Der \faStop\ Button wird gedrückt.}
{Die Analyse pausiert im aktuellen Zustand.}

\teststep{Die Analyse wurde durchgeführt und befindet sich im Zustand Pause.}
{Der Button \faForward\ wird geklickt.}
{Der Graph-Panel zeigt den Zustand der Analyse zu Beginn der nächsten Zeile an. Die Farben der Blöcke ändern sich, sowie auch das \glqq in \grqq\ und \glqq out \grqq\ der rechten Informationsleiste.}

\teststep{[opt]Die Analyse wurde durchgeführt und befindet sich im Zustand Pause.}
{Der Button \faFastForward\ wird geklickt.}
{Der Graph-Panel zeigt den Zustand der Analyse zu Beginn des nächsten Blocks an. Die Farben der Blöcke ändern sich, sowie auch das \glqq in \grqq\ und \glqq out \grqq\ der rechten Informationsleiste.}

\teststep{Die Analyse wurde durchgeführt und befindet sich im Zustand Pause.}
{Es wird der Button \faBackward\ geklickt.}
{Es wird der Zustand der Analyse in der vorhergehende Zeile angezeigt. Die Farben der Blöcke und die Informationen in der rechten Leiste zeigen wieder diesen Zustand an.}

\teststep{[opt]Die Analyse wurde durchgeführt und befindet sich im Zustand Pause.}
{Der Button \faFastBackward\ wird geklickt.}
{Es wird der Zustand der Analyse zu Beginn des aktuellen Blocks angezeigt oder der Beginn des letzten Blocks, falls derzeit die erste Zeile eines Blocks dargestellt wird. Gibt es keinen vorherigen Block wird der Startzustand angezeigt.}

\teststep{Die Analyse wurde durchgeführt und die Darstellung befindet sich eine Zeile vor dem Fixpunkt.}
{Der Button \faForward\ oder \faFastForward\ wird geklickt.}
{Es wird der Zustand der Analyse im Fixpunkt angezeigt und es wird eine Meldung gegeben, die über das Erreichen des Fixpunktes informiert.}

\test{Interaktion mit Graphen während Analyse}{tst:interaction}
\tests{}

\teststep{Die Analyse wurde an einer beliebigen Stelle angehalten.}
{Der Benutzer klickt im Graphen auf einen bestimmten Block.}
{Der \glqq in \grqq\ und \glqq out \grqq\ State zum aktuellen Analysezeitpunkt des gewählten Blocks wird in der Informationsleiste rechts eingeblendet.}

\teststep{Die Analyse wurde an einer beliebigen Stelle angehalten.}
{Der Benutzer klickt im Graphen auf eine bestimmte Zeile.}
{Der \glqq in \grqq\ und \glqq out \grqq\ State zum aktuellen Analysezeitpunkt der gewählten Zeile und des zugehörigen Blocks wird in der Informationsleiste rechts eingeblendet.}
	%note: don't split this document up with include{...}

\section{GUI-Entwürfe}

\begin{figure}[H]
\caption{GUI vor Ausführung der Analyse}
\centering
\includegraphics[angle=90, width=0.8\textwidth]{empty}
\end{figure}

\begin{figure}[H]
\caption{GUI während Ausführung der Analyse}
\centering
\includegraphics[angle=90, width=0.93\textwidth]{running}
\end{figure}
	
	% if you want to have the full glossary, not only the mentioned stuff
	% \glsaddall 
	
	\printglossaries
	
	
\end{document}