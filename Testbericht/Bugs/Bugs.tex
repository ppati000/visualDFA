\part{Bugs und Verbesserungen}

\section{dfa.framework}
\subsection{Bugs}

\begin{itemize}
	\item Vorberechnung einer \inlinecode{DFAExecution} geriet in eine Endlosschleife, falls der CFG leere Blöcke enthielt.
	Grund dafür war, dass ein leerer Block besonders behandelt wurde und in diesem Fall der Block für die nächste Iteration nicht aktualisiert wurde.
	\item Vorberechnung einer \inlinecode{DFAExecution} betrachtete nicht alle Blöcke, wenn sich Ausgangszustände von Beginn an nicht änderten. Grund dafür war, dass Blöcke nur auf die \inlinecode{Worklist} gesetzt wurden, wenn sich der Ausgangszustand eines Vorgängers geändert hatte. 
	\item Künstlicher Endblock im \inlinecode{SimpleBlockGraph} hatte keine Vorgänger. Grund dafür war, dass nach Erstellen des künstlichen Endblocks die vorherigen Endblöcke nicht als Vorgänger des neuen künstlichen Endblocks gesetzt wurden.
\end{itemize}

\section{dfa.analyses}
\subsection{Bugs}

\begin{itemize}
	\item \inlinecode{Locals}, die aufgrund ihres Typs ignoriert werden sollten, gelangten im Laufe einer Analyse doch in die \inlinecode{ConstantFoldingElement}s und \inlinecode{ConstantBitsElement}s. Grund dafür war, dass bei Zuweisungen der Typ der Local nicht überprüft wurde. Dies wurde durch Einführung einer Methode \inlinecode{isLocalTypeAccepted}\inlinecode{(Type t):boolean} behoben.
	\item Der \inlinecode{TaintJoin} lieferte auf einem einzelnen \inlinecode{TaintElement} immer \inlinecode{BOTTOM} zurück. Grund dafür war die falsche Initialisierung des Referenzelements, mit dem alle außer dem ersten Element verglichen wurden.
	\item Casts führten zu Abstürzen bei der \inlinecode{ConstantFoldingTransition}. Grund dafür war, dass Casts versehentlich ignoriert wurden.
\end{itemize}

\section{gui.visualgraph}

\subsection{Bugs}

\begin{itemize}
  \item Das Auswählen von Blöcken und Zeilen im Graphen war fehlerhaft, sodass falsche Blöcke und Zeilen ausgewählt wurden.
        Grund dafür war eine Veränderung der Variable \inlinecode{LABEL_INSET} in JGraphX, welche den Text der Zeilen um die gesetzte Pixelanzahl einrücken sollte.
        Ein JGraphX-Bug führte dabei aber auch zu beschriebener Nebenwirkung.
        Dies wurde dadurch behoben, die Variable nicht mehr zu verändern und stattdessen die einzelnen Zeilen-\inlinecode{mxCell}s manuell einzurücken.
  \item Der Text in den Zeilen war nicht korrekt vertikal zentriert.
        Dieses Fehlverhalten trat ebenfalls wegen eines JGraphX-Bugs auf und wurde durch den Aufruf der Methode \inlinecode{mxGraph.setHtmlLabels(true)} verhindert.
  \item In exportierten Bildern des Graphen wurde der aktuell ausgewählte Block nicht farblich markiert.
        Dieses Problem wurde ebenfalls durch die entsprechende Implementierung in JGraphX verursacht.
        Zur Umgehung wurde ein Workaround angewendet, der dem jeweils aktiven Block einen Border hinzufügt, den Graphen exportiert und anschließend den Border wieder zurücksetzt.
        Dies bleibt für den Benutzer unbemerkt und stellt sicher, dass in den exportierten Bildern keine Information verloren geht.
  \item Der Graph war weiterhin sichtbar, nachdem der Benutzer den Stop-Button betätigt hatte.
        Dies war ein Rendering-Problem und wurde durch den Aufruf von \inlinecode{repaint()} behoben.
\end{itemize}

\subsection{Verbesserungen}
\begin{itemize}
  \item Die Methode \inlinecode{getBlockAndLineNumbers()} in der Klasse \inlinecode{UIAbstractBlock} sowie deren Subklassen wurde in zwei entsprechende Methoden aufgespaltet.
  \item Die Methode \inlinecode{renderGraph()} in der Klasse VisualGraphPanel nimmt nun keinen \inlinecode{isFirstRender}-Parameter mehr entgegen, da der entsprechende Zustand nun intern verwaltet wird.
  \item Einige platzintensive Code-Elemente wie \inlinecode{dynamicinvoke} und \inlinecode{goto}-Statements werden nun im Graphen abgekürzt, um den Text leichter lesbar zu machen.
  \item Der Batch-Graph-Export benutzt nun einen eigenen Thread und bietet dem Benutzer eine Fortschrittsanzeige, welche sich mit jedem Schritt des Exports füllt.
        Dabei wird dem Benutzer auch angezeigt, in welchen Pfad die Bilder exportiert werden.
        Während des Exports kann der Benutzer das Programm normal weiterverwenden; die Fortschrittsanzeige bleibt dabei sichtbar.
        
        Nachdem diese Fortschrittsanzeige implementiert wurde, fiel im Rahmen genereller Tests auf, dass die Benutzerführung nun inkonsistent war, da beim Einzelexport keine Fortschrittsanzeige verwendet wurde.
        Da ein Einzelexport zu schnell vonstattengeht, um einen tatsächlichen Fortschritt anzeigen zu können, wird hier nun eine Fortschrittsanzeige verwendet, die sich nicht auf den tatsächlichen Fortschritt bezieht und dem Benutzer lediglich Konsistenz mit dem Batch-Export bietet.
\end{itemize}

\section{CodeProcessor}

Im Modul \inlinecode{CodeProcessor} haben sich die folgenden Änderungen ergeben: \newline
Zunächst sind die Klasse \inlinecode{NoFilter} und \inlinecode{StandardFilter} weggefallen und nur die Oberklasse \inlinecode{Filter} wurde erhalten. Die Klasse \inlinecode{StandardFilter} war zunächst dafür gedacht, vererbte Methoden und Java-Standard-Methoden aus der Menge der eingegeben Methoden herauszufiltern. Dies hat sich jedoch als nicht nötig erwiesen, da die Menge der vererbten Methoden gering ist, da immer nur eine Klasse analysiert wird. Des Weiteren werden Java-Standard-Methoden, die nicht eigenhändig überschrieben wurden, durch die Prozessierung mit Soot gar nicht erst ausgegeben. Die Klasse \inlinecode{Filter} ist dafür verantwortlich, die hinzugefügten Methoden für die Taint-Analyse wieder zu entfernen und entsprechende Tags zu setzen. \newline
Im Modul \inlinecode{CodeProcessor} wurden außerdem Änderungen in der \inlinecode{CodeProcessor}-Klasse vollzogen, die dazu führen, dass bei Verwendung der Taint-Analyse die Methode \inlinecode{sensitive(Datentyp x)} auf alle verwendeten Variablen einzeln angewendet werden kann.\newline
Ein gefundener und behobener Bug war ein Compilererror, der sich bei Eingabe von Code der zum Beispiel \textit{String s = "text";} beinhaltete ergab. Dieser Fehler wurde durch eine Regex hervorgerufen, die verantwortlich dafür war, Kommentare aus dem zu kompilierenden Code zu entfernen. Diese entfernte unglücklicherweise auch Text im Code, sodass aus dem oben genannten Codestück \textit{String s = ;} resultierte. Dieser Bug konnte durch die Verwendung von getrennten Regexes für mehrzeilige und einzeilige Kommentare behoben werden.\newline
Weitere Änderungen in diesem Modul ergeben sich aus der Verbesserung der Benutzerfreundlichkeit in Bezug auf den notwendigen Java Compiler. Ist das Programm korrekt gestartet ist auch sichergestellt, dass ein funktionierender Compiler vorhanden ist. Diese Verbesserung ergibt sich durch Änderungen im \inlinecode{Controller}-Modul, die deswegen dort beschrieben werden.\newline

\section{Controller}

Im Modul \inlinecode{Controller} haben sich die folgenden Änderungen ergeben: \newline
Die \inlinecode{Controller}-Klasse legt zentral den Pfad fest, unter welchem Dateien, die das Programm generiert, gespeichert werden, wie etwa Bilder des Graphexports, gespeicherte .class-Dateien oder die zentrale Options.txt-Datei. Diese Dateien werden im Home-Verzeichnis des Nutzers im Ordner \textit{visualDfa} gespeichert, um sicherzustellen, dass das laufende Programm ausreichende Schreibrechte besitzt. Des Weiteren legt die \inlinecode{Controller}-Klasse den Pfad der dynamisch geladenen Analysen fest. Diese liegen in einer Ordnerstruktur der Form \textit{.\textbackslash visualDfa\textbackslash analyses}, wobei sich der Ordner \textit{visualDfa} im gleichen Ordner wie die aktuell ausgeführte .jar-Datei befindet. \newline
Die Options.txt-Datei enthält zentrale Einstellungen des Programms. Derzeit sind dies der JDK-Pfad und eine Variable, welche festlegt, ob eine Warnung beim Stoppen der Analyse ausgegeben werden soll oder nicht. Zur besseren Umsetzung wurde die \inlinecode{OptionFileParser}-Klasse hinzugefügt, was die größte Änderung dieses Moduls darstellt. Der Aufbau der Klasse ist im Folgenden dargestellt. Die Aufgabe der Klasse ist das Einlesen dieser Datei, die Überprüfung der Korrektheit der Angaben und die Umsetzung der Einstellungen im Programmcode. Beim ersten Start des Programms wird eine solche Datei einmalig erstellt und bei folgenden Programmstarts immer wieder ausgelesen. Somit wird der Benutzer auch nur einmal nach seinem JDK-Pfad gefragt. Diese Art der Verarbeitung erleichtert es dem Benutzer auch noch weitere Einstellungen in die Options.txt-Datei aufzunehmen und damit spezielle Funktionen des Programms zu steuern.\newline

\begin{itemize}
	\item \textbf{public OptionFileParser (String outputPath, ProgramFrame programFrame):} Der Konstruktor dieser Klasse wird direkt beim Programmstart aufgerufen und liest, wie oben beschrieben den Inhalt der Options.txt-Datei ein oder erstellt eine neue Datei. Dabei wird der Nutzer im Fall eines falsch gesetzten Java-Homes mittels eines FileChoosers nach dem Pfad zu seiner JDK gefragt.
	\item \textbf{public void setShowBox(boolean shouldShowBox):} Methode, welche aufgerufen wird, wenn in der \inlinecode{GenericBox} der Haken bei \textit{Don't show again} gesetzt wird. Die Options.txt-Datei wird als Folge davon mit den gewählten Einstellungen neu in den entsprechenden Pfad geschrieben. Damit wird diese Information für die folgenden Programmstarts gespeichert.
	\item \textbf{public boolean shouldShowBox():} Methode die zurück gibt, ob der Nutzer die entsprechende Warnung beim Stoppen der Analyse sehen möchte oder ausgewählt hat, dass sie nicht mehr dargestellt werden soll. 
\end{itemize}