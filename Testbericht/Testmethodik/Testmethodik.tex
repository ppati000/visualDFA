\part{Testmethodik}

Zunächst wurden einzelne Klassen weitestgehend unabhängig voneinander in Unit-Tests getestet.
Für Klassen, die viele Abhängigkeiten zu anderen komplexen Klassen hatten, wurde Mockito benutzt, um Mock-Objekte zu erzeugen, die statt der benötigten komplexen Objekte verwendet werden konnten.
Die einzelnen Module wurden jeweils von den Personen getestet, die auch für deren Implementierung verantwortlich waren:

\begin{tabular}{l*{6}{c}r}
	Modul & Person(en) \\
	\hline
	\inlinecode{dfa.framework} & Nils, Sebastian \\
	\inlinecode{dfa.analyses} & Nils, Sebastian \\
	\inlinecode{controller} & Anika \\
	\inlinecode{codeprocessor} & Anika \\
	\inlinecode{gui} & Michael \\
	\inlinecode{gui.visualgraph} & Patrick \\
\end{tabular}


\section {Benutzertest}
Beim Benutzertest wurden mehrere Tester, welche mit dem Themengebiet der Informatik vertraut sind dazu aufgefordert, das Programm unter Anleitung eines Programmierers zu benutzen und Rückmeldung über Benutzerfreundlichkeit und eventuell aufgetretene Bugs zu geben. 
Aufgrund dieser Rückmeldungen wurden Teile der Benutzeroberfläche überarbeitet, um das Programm für neue Benutzer intuitiver zu gestalten.
Dazu zählt zum Beispiel Änderungen an der Größe und Farbe der verschiedenen Arbeitsbereiche, um die Benutzerführung beim Start des Programmes zu verbessern.
Außerdem wurden kleinere Fehler in der Programmlogik gefunden, welche keine wirklichen \enquote{Bugs} waren, aber die Benutzbarkeit des Programmes eingeschränkt hätten.
Dazu zählt zum Beispiel, dass der Benutzer beim Auswählen einer Datei mit Java-Code nur Dateien mit einer .java-Endung angezeigt bekam. Nicht hingegen konnte er Ordner, oder andere Dateitypen, in welchen möglicherweise Code gespeichert werden könnte, sehen.
Diese Tests bezogen sich hauptsächlich auf Teile der Grafischen Benutzeroberfläche, welche mit den standardmäßigen Test-Frameworks nur schwer sinnvoll getestet werden konnte.