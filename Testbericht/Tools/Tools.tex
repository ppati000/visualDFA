\part{Tools}
Um effektiv zu Testen, wurden einige Tools eingesetzt. Dies sind zwar weitestgehend Standart-Tools für das Testen von Java-Code, dennoch werden diese hier kurz vorgestellt.

\section{JUnit}
JUnit ist ein Test-Framework für Java.
Damit können Tests realisiert werden, die automatisch ausführbar sind.
Die erwarteten Resultate können mittels Assertions festgelegt werden, die fehlschlagen, wenn die tatsächlichen Resultate von den Erwarteten abweichen.
JUnit kann alle Tests einer bestimmten Klasse, eines Packages, oder alle vorhandenen Tests im Projekt ausführen und erstellt dann eine Übersicht über die erfolgreichen und fehlgeschlagenen Tests.
Für fehlgeschlagene Tests wird präzise angezeigt, welche Assertion fehlgeschlagen ist und für auftretende Exceptions wird ein vollständiger Stacktrace ausgegeben.
Dies ermöglicht komfortables Testen und relativ einfaches Debuggen.
Außerdem lässt sich durch die automatischen Tests zuverlässig verhindern, dass einmal behobene Bugs wieder unbemerkt auftreten. 

\section{Mockito}
\todo{@Patrick}

\section{EclEmma}
EclEmma ist ein Eclipse-Plugin zum Messen der Code-Coverage.
EclEmma basiert auf dem Code-Coverage-Tool JaCoCo, dem Nachfolger von EMMA.
Mittels EclEmma lässt sich die Code-Coverage präzise und komfortabel messen:
Die Coverage-Statistik wird nach Klassen und Packages aufgeschlüsselt dargestellt.
Sehr hilfreich ist auch, dass getesteter Code je nach Coverage-Status eingefärbt wird: vollständig getestete Zeilen werden grün, teilweise getestete Zeilen gelb und nicht getestete Zeilen rot eingefärbt.
Dies ermöglicht einen präzisen Überblick darüber, was genau getestet wurde.





